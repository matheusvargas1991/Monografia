% ---------------------------------------------------------------
% ---------------------------------------------------------------
% Modelo de Trabalho Acadêmico utilizando classe repUERJ para
% elaboração de teses, dissertação e trabalhos monográficos em
% geral.
%
% Este arquivo está editado na codificação de caracteres UTF-8.
%
% As referencia estão baseadas no modelo bibtex e citação em
% autor-data
%
% Este modelo foi criado por 
% 	Dr. Luís Fernando de Oliveira
% 	Professor Adjunto do Dep. de Física Aplicada e Termodinâmica
% 	Instituto de Física Armando Dias Tavares
% 	Universidade do Estado do Rio de Janeiro - UERJ
%
% A classe repUERJ.cls foi criada a partir do código original 
% disponibilizado pelo grupo CódigoLivre (coordenado por
% Gerald Weber). Foram feitas adequações para implementação das 
% normas de elaboração de teses e dissertações da UERJ.
%
% Os estilos repUERJformat.sty codificam os elementos
% pré-textuais e pós-textuais.
%
% O estilo repUERJpseudocode.sty codifica a elaboração de
% algoritmos utilizando um glossário desenvolvido por mim
% (Luís Fernando), o mesmo usado em meu curso de Física
% Computacional.
%
% Todo este material está disponível também no meu site
%      http://sites.google.com/site/deoliveiralf
%
% As normas da UERJ para elaboração de teses e dissertações 
% pode ser obtidas no documento disponível no site
%      http://www.bdtd.uerj.br/roteiro_uerj_web.pdf
%
% Agradecimentos ao NPROTEC/Rede Sirius/UERJ e à Biblioteca
% Setorial da Física.
% ---------------------------------------------------------------
% ---------------------------------------------------------------
%
% Adaptado para o Departamento de Eng. de Sistemas e Computação pelo
% professor João Araujo
\documentclass[a4paper,12pt,oneside,onecolumn,final,fleqn]{repUERJ}
% ---
% Pacotes fundamentais 
% ---
\usepackage[brazil]{babel}  % adequação para o português Brasil
\usepackage[utf8]{inputenc} % Determina a codificação utilizada
% (conversão automática dos acentos)
\usepackage{makeidx}        % Cria o índice
\usepackage{hyperref}       % Controla a formação do índice
\usepackage{indentfirst}    % Indenta o primeiro paragrafo de
% cada seção.
\usepackage{graphicx}       % Inclusão de gráficos
\usepackage{subfig}
\usepackage{multirow}
\usepackage{amsmath,amssymb}  % pacotes matemáticos
\usepackage[alf]{abntex2cite} % pacote para citacoes
\usepackage[font=default,frame=no]{repUERJformat} % pacote para as 
% normas da UERJ
% ---
% pacote auxiliar para elaboração de pseudocódigos
% este pacote pode ser retirado caso nao se planeje
% elaborar pseudocódigos
% ---
\usepackage[dots=yes]{repUERJpseudocode}
\usepackage{listings}
\lstset{language=XML, 
	keywordstyle=\color{blue},
	basicstyle=\small,
	showstringspaces=false,
	xleftmargin=0pt,
	extendedchars=true,
	inputencoding=utf8,
	frame=none,
	breaklines=true,
	emph={True,False},
	texcl=true,
	literate={á}{{\'a}}1 {à}{{\`a}}1 {ã}{{\~a}}1 {â}{{\^a}}1 {é}{{\'e}}1 {í}{{\'i}}1 {ó}{{\'o}}1 {ú}{{\'u}}1 
	{ê}{{\^e}}1 {é}{{\'e}}1 {ô}{{\^o}}1 {õ}{{\~o}}1 {ç}{{\c{c}}}1 {º}{${^\circ}$}1,, 
	commentstyle=\color{red},
	tabsize=2,
	stringstyle=\color{red},
	columns = fixed
} 

\usepackage[maxfloats=25]{morefloats}
\usepackage{array}
\usepackage{multirow}
\setlength\extrarowheight{2pt}

% ***************************************************************
% Informações de autoria e institucionais
% ***************************************************************

%----------------------------------------------------------------
% Imagens pretextuais (precisam estar no mesmo diretório deste arquivo .tex)
%----------------------------------------------------------------

\logo{logo_uerj_cinza.png}
\marcadagua{marcadagua_uerj_cinza.png}{1}{160}{255}

%----------------------------------------------------------------
% Informações da instituição
%----------------------------------------------------------------

\instituicao{Universidade do Estado do Rio de Janeiro}
{Centro de Tecnologia e Ciências}  
{Faculdade de Engenharia}
{Departamento de Sistemas e Computação} 

%----------------------------------------------------------------
% Informações da autoria do documento
%----------------------------------------------------------------

% \oautor{Nóme}{Sóbrenome}
%       {Iniciais do nome} % iniciais do nome

\autor{Matheus}{Vargas}
{M.V.} % iniciais do nome

\titulo{Adaptação do Software E-foto para Atendimento aos Requisitos de Pacotes Debian-gis}
\title{Adaptation of e-foto software to meet the requirements of debian-gis packages}

% se não for usar a quarta palavra chave, deixar o campo vazio: {}
\palavraschaves{Primeira palavra-chave}
{Segunda palavra-chave}
{Terceira palavra-chave}
{Quarta palavra-chave (opcional) ou vazio}

\keywords{First keyword}
{Second keyword}
{Third keyword}
{Fourth keyword or empty}

\orientador{Prof. Dr.} 
{João}{Araujo Ribeiro} 
{Faculdade de Engenharia -- UERJ} 


%----------------------------------------------------------------
% Grau pretendido (Doutor, Mestre, Bacharel, Licenciado) e Curso
%----------------------------------------------------------------

\grau{Engenheiro} % Doutor, Mestre, Bacharel, Licenciado
\curso{Graduação}
%\areadeconcentracao{linha de pesquisa} % opcional

%----------------------------------------------------------------
% Informações adicionais (local, data e paginas)
%----------------------------------------------------------------

\local{Rio de Janeiro} 
\data{dd}{mês}{2021} 

% ***************************************************************
% Configurações de aparência do PDF final
% ***************************************************************

% alterando o aspecto da cor azul
%\definecolor{blue}{RGB}{41,5,195}
%\definecolor{apricot}{RGB}{251,206,177}

% informações do PDF
\hypersetup{
	unicode=false,
	pdftitle={\UERJtitulo},
	pdfauthor={\UERJautor},
	pdfsubject={\UERJpreambulo},
	pdfkeywords={PALAVRAS-CHAVES:}{ \chaveA}{ \chaveB}{ \chaveC}{ \chaveD},
	pdfproducer={\packagename}, % producer of the document
	colorlinks=true,   % false: boxed links; true: colored links
	linkcolor=black,   % color of internal links blue
	citecolor=black,   % color of links to bibliography blue
	filecolor=black,   % color of file links magenta
	urlcolor=black,
	bookmarksdepth=4,
	%   backref=true,
	%   pagebackref=true,
	%   bookmarks=true,
}

% ***************************************************************
% Índice remissivo
% ***************************************************************
%
\makeindex % compila o índice; se não for usar, comentar
%
% ***************************************************************
% Fim do preâmbulo
% ***************************************************************

%/\/\/\/\/\/\/\/\/\/\/\/\/\/\/\/\/\/\/\/\/\/\/\/\/\/\/\/\/\/
\begin{document}
	%/\/\/\/\/\/\/\/\/\/\/\/\/\/\/\/\/\/\/\/\/\/\/\/\/\/\/\/\/\/
	
	%XXXXXXXXXXXXXXXXXXXXXXXXXXXXXXXXXXXXXXXXXXXXXXXXXXXXXXXXXXXXXXXX
	% ELEMENTOS PRE-TEXTUAIS
	%XXXXXXXXXXXXXXXXXXXXXXXXXXXXXXXXXXXXXXXXXXXXXXXXXXXXXXXXXXXXXXXX
	\frontmatter % inicia a área dos elementos pré-textuais
	%XXXXXXXXXXXXXXXXXXXXXXXXXXXXXXXXXXXXXXXXXXXXXXXXXXXXXXXXXXXXXXXX
	
	% ----------------------------------------------------------
	% Capa e a folha de rosto
	% ----------------------------------------------------------
	%
	\capa
	\folhaderosto
	%
	% ----------------------------------------------------------
	% Inserir a ficha catalográfica
	% ----------------------------------------------------------
	% ---
	% A biblioteca deverá providenciar a ficha catalográfica. 
	% Salve a ficha no formato PDF. Use o nome do arquivo PDF 
	% como argumento do comando. 
	% Exemplo: ficha catalográfica no arquivo 'ficha.pdf'
	%     \fichacatalografica{ficha.pdf}
	%
	% Enquanto não possuir a ficha catalográfica, use o comando sem
	% argumentos.
	% ---
	%
	\fichacatalografica{}
	%
	% ----------------------------------------------------------
	% Folha de aprovação
	% ----------------------------------------------------------
	%
	\begin{folhadeaprovacao}
		\assinatura{Cargo Título Nome Completo}
		{Unidade -- Instituição}
		\assinatura{Cargo Título Nome Completo}
		{Unidade -- Instituição}
		\assinatura{Cargo Título Nome Completo}
		{Unidade -- Instituição}
		\assinatura{Cargo Título Nome Completo}
		{\UERJunidade \UERJunidadenome\ -- UERJ}
	\end{folhadeaprovacao}
	% Agradecimentos
	% ----------------------------------------------------------
	%
	\pretextualchapter{Agradecimentos}
	
	Texto de agradecimento (opcional).
	%
	% ----------------------------------------------------------
	% Epigrafe (opcional)
	% ----------------------------------------------------------
	%
	\pretextualchapter{}
	\vfill
	%
	% ----------------------------------------------------------
	% RESUMO
	% ----------------------------------------------------------
	%
	\pretextualchapter{Resumo}
	\referencia % linha em branco depois
	
	Texto do resumo em português.
	~\\
	
	\imprimirchaves % linha em branco antes
	%
	% ----------------------------------------------------------
	% Abstract
	% ----------------------------------------------------------
	%
	\pretextualchapter{Abstract}
	\reference % linha em branco depois
	
	Texto do resumo em inglês.
	~\\
	
	\printkeys % linha em branco antes
	%
	% ----------------------------------------------------------
	% Listas de ilustrações e tabelas
	% ----------------------------------------------------------
	%
	\listadefiguras
	%
	% ----------------------------------------------------------
	% Lista de abreviaturas e siglas (opcional)
	% ----------------------------------------------------------
	%
	\pretextualchapter{Lista de abreviaturas e siglas}
	\abreviatura{sigla 1}{por extenso}
	\abreviatura{sigla 2}{por extenso}
	\abreviatura{sigla 3}{por extenso}
	%
	% ----------------------------------------------------------
	% ----------------------------------------------------------
	% Sumario
	% ----------------------------------------------------------
	%
	\sumario
	%
	%XXXXXXXXXXXXXXXXXXXXXXXXXXXXXXXXXXXXXXXXXXXXXXXXXXXXXXXXXXXXXXXX
	% ELEMENTOS TEXTUAIS
	%XXXXXXXXXXXXXXXXXXXXXXXXXXXXXXXXXXXXXXXXXXXXXXXXXXXXXXXXXXXXXXXX
	\mainmatter % inicia a área de desenvolvimento do conteúdo
	%XXXXXXXXXXXXXXXXXXXXXXXXXXXXXXXXXXXXXXXXXXXXXXXXXXXXXXXXXXXXXXXX
	
	
	
	%================================================================
\chapter*{Introdução}
%================================================================

O software E-foto está em desenvolvimento desde 2004 pelo laboratório de fotogrametria da Faculdade de Engenharia da Universidade do Estado do Rio de Janeiro, e tem como foco oferecer uma \textbf{Estação Fotogramétrica Digital} gratuita e livre, que poderia ajudar os alunos a entender os princípios por trás da fotogrametria, evitando assim custos elevados que teriam impedido muitos de aprender sobre o assunto. O usuário interessado tem a possibilidade de instalá-lo tanto em sistemas operacionais \textit{Linux} (preferencialmente \textit{Ubuntu}) quanto \textit{Windows}, porém, mesmo após o lançamento de várias versões do programa, o código nunca foi empacotado de maneira que ele pudesse ser aceito como pacote oficial nos repositórios \textit{Debian}, uma vez que para isso é necessário seguir uma série de normas que são explicadas na política de empacotamento do \textit{Debian}, e que envolvem desde licenças do código que garantem o fato do mesmo ser livre, até regras sobre o próprio empacotamento do software.

\section*{Objetivo}

Esse projeto tem como objetivo adaptar o código do software E-foto e gerar um pacote  de maneira que ele possa ser aceito oficialmente nos repositórios \textit{Debian} e a partir disso possa participar do \textit{Debian-gis}, que é uma parte do \textit{Debian} direcionada para os usuários de \textbf{Sistemas de Informação Geográfica}. Com isso o software terá sua instalação facilitada, pois pacotes do \textit{Debian} são instalados com comandos simples e também a sua divulgação aumentada, pois estará num ambiente direcionado para pessoas com interesse semelhante às suas funcionalidades. 

Além desse objetivo principal, também será apresentado no projeto informações como tutoriais de instalação do software E-foto no \textit{Ubuntu} e no \textit{Windows}, assim como um explicação sobre o uso de suas principais aplicações, e também um tutorial de como é realizado o empacotamento de maneira correta de acordo com a política do Debian e seus documentos oficiais.

\section*{Estrutura da Monografia}

Esta monografia está dividida em seis capítulos: \textbf{Introdução, efoto, Pacote debian efoto, Debian-gis, Resultados e Conclusão}.

No primeiro e atual capítulo é fornecida uma explicação sobre o projeto do Efoto, uma explicação sobre o \textit{Debian} e sobre o \textit{Debian-gis}, assim como a motivação para este projeto.

No capítulo \textbf{efoto} está disponível a explicação completa sobre como instalar o software em suas diferentes formas, a partir de um tutorial testado em sistemas \textit{Ubuntu} e em \textit{Windows}, desde a obtenção do código fonte até a explicação sobre suas funcionalidades dentro do programa.

No capítulo \textbf{Pacote debian efoto} é apresentado um passo a passo sobre como realizar o empacotamento do efoto de acordo com a política do \textit{Debian}, além  dos passos essenciais para tal. Esse capítulo tem como objetivo seguir o  que é um documento oficial com explicações sobre o empacotamento.

No capítulo \textbf{Debian-gis} é explicado o que é o \textit{Debian}, o que é o \textit{Debian-gis} e o que é necessário para que o pacote do Efoto seja aceito no \textit{Debian-gis} depois de empacotado corretamente.

No capítulo \textbf{Resultados} são apresentados os resultados dos testes realizados de instalação e funcionamento do pacote efoto em diferentes sistemas operacionais.

O sexto e último capítulo apresenta um resumo sobre o foi desenvolvido, ideias para seu uso e aplicações além de sugestões de melhorias futuras.

	%================================================================
\chapter{O programa E-foto}
%================================================================

Nesse capítulo será apresentado o software E-foto, suas funcionalidades e o que deve ser feito para sua obtenção e instalação. Sendo explicado que pacotes serão necessários para o funcionamento do programa após e realização do seu download.
%fiz uma introduçaõ genérica, que deve ser modificada quando o projeto for mais completo

\section{O que é o E-foto}

O E-foto é um software live para fotogrametria digital que é desenvolvido pelo laboratório de fotogrametria da Universidade do Estado do Rio de Janeiro desde 2004 e tem como objetivo, além da criação de um software inteiramente funcional (uma estação fotogramétrica gratuita), levar aos alunos o conhecimento de forma gratuita sobre fotogrametria digital, sendo de forma didática ou ate mesmo na prática por meio de acesso ao código, uso da plataforma e até criação de novos módulos para o software. A fotogrametria é a obtenção de informações confiáveis por meio de imagens obtidas por sensores, sendo no caso da fotogrametria digital a entrada de dados são imagens digitais provenientes de máquinas digitais ou pela digitalização de imagens obtidas de forma analógica.

\section{Instalação do E-foto}
\subsection{Em Sistemas Linux}
\subsection{Em Sistemas Windows}

\subsection{Compilação do E-foto a partir dos fontes}
\subsection{Em Sistemas Linux}
% é necessário que você faça um verdadeiro passo a passo.
% detalhe cada passo. Pense que você está ensinando o seu pai a compilar o e-foto. Seja minucioso e sem ambiguidades. 

\subsubsection{Passo 1 - Baixar código fonte}
Primeiramente, para realizar a instalação do software E-foto, o usuário deve fazer o download do seu código fonte e para isso é necessário a instalação do subversion, se já não estiver instalado. Para tal, o usuário deve abrir o terminal do seu sistema Linux, que pode ser feito digitando terminal na barra de busca ao apertar a tecla do Windows no seu teclado ou pelo atalho do teclado \texttt{ctrl + alt + T}. Com o terminal aberto, o usuário começará a instalação do SVN com os comandos: 
\begin{lstlisting}[language=bash]
	$ sudo apt update
	$ sudo apt install subversion
\end{lstlisting}
% (que atualizará o gerenciador de pacotes)
 % e sudo apt install apache2 apache2-utils 
 %esses pacotes só são necessários se for criar um servidor svn, o que não é caso. Para baixar o e-foto, ou qualquer software por svn, basta instalar o subversion, sem o apache2. Veja em https://askubuntu.com/questions/258151/how-can-i-install-subversion-client-in-ubuntu
 %(que instalará os pacotes do apache2, onde Apache licencia um sistema de controle de versão de código aberto). Logo após execute o comando sudo apt-get install subversion libapache2-mod-svn subversion-tools libsvn-dev (que instalará o SVN e todas as dependências necessárias). Agora é preciso habilitar os módulos apache2 para o SVN funcionar e isso é feito com os comandos sudo a2enmod dav e sudo a2enmod dav\_svn. Então o usuário tem que reiniciar o Apache com o comando sudo service apache2 restart. O próximo passo é criar uma conta de usuário SVN que é feito com o comando sudo htpasswd -cm /etc/apache2/dav\_svn.passwd admin (nesse passo o usuário precisará fornecer sua senha de administrador). E o último passo de configuração do SVN é reiniciá-lo com o comando sudo systemctl restart apache2.service. 
 
Com o SVN instalado e configurado, basta o usuário digitar no terminal o comando:
 
\begin{lstlisting}[language=bash]
 $ svn checkout https://svn.code.sf.net/p/e-foto/code
\end{lstlisting}

Com a utilização desse comando o download de todo o código fonte do software E-foto será feito automaticamente.  
    
\subsubsection{Passo 2 - Instalar pacotes necessários à compilação}  
Após a realização do download do código fonte do E-foto, o usuário deve ficar atento aos pacotes necessários em seu ambiente para que o software E-foto possa ser instalado e ter seu funcionamento sem erros. Para a instalação dos pacotes o usuário deve buscar abrir novamente o terminal. Os pacotes necessários para instalação do e-foto são:
\begin{itemize}
   	\item libgdal.dev
   	\item build-essential
   	\item libfontconfig1
   	\item mesa-common-dev
   	\item libx11-xcb-dev
   	\item libglu1-mesa-dev
\end{itemize}
Cada pacote deve ser instalado com respectivamente com os seguintes comandos:	

\begin{lstlisting}[language=bash]
	$ sudo apt install libgdal-dev
	$ sudo apt install build-essential
	$ sudo apt install libfontconfig1
	$ sudo apt install mesa-common-dev
	$ sudo apt install libx11-xcb-dev 
	$ sudo apt install libglu1-mesa-dev
\end{lstlisting}				
	
O pacote libgdal.dev é o contém as funcionalidades da GDAL, onde GDAL é uma biblioteca de tradução para formatos geoespaciais. O pacote libfontconfig1 contém uma biblioteca projetada para achar fontes no sistema e selecioná-las de acordo com os requisitos especificados pelas aplicações, e o usuário deve instalar o driver XCB e o OpenGl através dos pacotes mesa-common-dev, libx11-xcb-dev e libglu1-mesa-dev. 
    
\subsection{Passo 3 - Instalar os pacotes de instalação do Qt 5}   
A instalação do Qt 5 via terminal deve ser feita através dos seguintes pacotes:
\begin{itemize}
	\item qt5-default
	\item qt5-qmake
\end{itemize}   
Esses pacotes devem ser instalados através dos seguintes comandos do terminal:
   
\begin{lstlisting}[language=bash]
	$ sudo apt install qt5-default
    $ sudo apt install -y qt5-qmake
\end{lstlisting}	
    
Após esses procedimentos o usuário ja terá um ambiente de compilação pronto para a instalação do E-foto, assim como o plataforma em que o mesmo foi desenvolvido.
    
\subsection{Passo 4 - Compilar e executar o software E-foto}
Para compilar e executar o código do E-foto via terminal, após a realização de todos os passos necessários, o usuário deve utilizar os seguintes comandos no terminal:
   
\begin{lstlisting}[language=bash]
   	$ cd diretório/
   	$ qmake e-foto.pro
   	$ make
   	$ ./build/bin/e-foto
\end{lstlisting}
   
O comando cd diretório/ servirá para o usuário percorrer o caminho até o diretório onde está o arquivo e-foto.pro (atualmente code/branches/e-foto-trunk-candidate), depois o qmake e o make irão realizar a compilação e gerar o executável do E-foto. Por fim o último comando irá executar o software E-foto.
   
   % Por último os pacotes para a instalação do Qt 5 propriamente dito que são os pacotes qt5-default, instalado através do comando no terminal sudo apt install qt5-default, e qt5-qmake, instalado através do comando sudo apt install -y qt5-qmake, que permitirão o uso do Qt 5 e consequentemente a instalação do Software E-foto

\subsection{Em Sistemas Windows}

Em sistemas Windows o usuário deve se certificar de estar utilizando um sistema de 64-bits pois não existe versão do E-foto para sistemas de 32-bits, para realizar esse procedimento de verificação de qual é a versão do sistema que está sendo utilizada basta digitar "informações do sistema" no menu iniciar, no menu Resumo do Sistema existe a opção Versão do Sistema que dará ao usuário essa informação. O próximo passo é o download dos pacotes binários do Gdal no Windows que pode ser realizado através desse link https://repo.msys2.org/distrib/x86\_64/msys2-x86\_64-20200720.exe, % sublinhas têm quie ser colocadas depois do caracter de escape \, senão o latex tenta interpŕetar como comando
  após a o download e da instalação do MSYS o usuário deve executar o que vai abrir o terminal do próprio MSYS que contém uma versão portada do gerenciador de pacotes Pacman, nessa ultima versão do Pacman deve ser digitado no terminal o código pacman -Syuu, que gerará um série de instruções a serem seguidas até que o usuário possa repetir o código e receber a mensagem de que nada necessita ser atualizado. Com o ambiente atualizado o usuário deve digitar o seguinte comando pacman -S mingw64/mingw-w64-x86\_64-gdal, que realizará finalmente o binário da GDAL. Após todo esse procedimento o usuário pode consultar no terminal do MSYS através do código gdalinfo --version se o gdal realmente foi instalado e a versão em que ele está disponível. Para seguir, o usuário deve possuir o Tortoise SVN que pode ser adquirido gratuitamente por esse link https://tortoisesvn.net/downloads.html, após a instalação do SVN o usuário deve buscar no programa a opção de SVN Checkout e no espaço para colar uma URL colocar https://svn.code.sf.net/p/e-foto/code, que deve gerar o download de todo o código fonte do E-foto. O Usuário deve, por ultimo, realizar o download do Qt 5 e do Qtcreator no website https://www.qt.io/download, deixando claro que na parte da instalação deve ser instalado apenas o mingw 64-bits, e o Qt 5 referente a esse sistema. Com os arquivos do código fonte do E-foto disponíveis o usuário deve procurar no caminho e-foto-code/branches/e-foto-trunk-candidate por um arquivo chamado e-foto.pro. Quando abrir o projeto no Qtcreator será necessário configurar o que deverá ser usado, começando pela escolha do kit que será usado que é o Qt 5 e o mingw 64-bits. Após o projeto estar aberto, o usuário deve ir na guia project e desmarcar a opção shadow build, e depois procurar na aba build a opção build E-foto, quando acabar a compilação é só buscar a opção Run e o E-foto vai abrir pronto para o uso.
	%================================================================
\chapter{Debian-GIS}
%================================================================
\section{O que é o Debian-GIS}

O Debian-GIS é um Debian Pure Blend, o que significa ser uma parte do Debian onde seus pacotes estão agrupados e disponíveis para um determinado grupo de usuários que possuem necessidades especiais tornando desnecessário o acesso a todos os pacotes Debian disponíveis. O Debian-GIS tem como objetivo atender os usuários interessados em um Sistema de Informação Geográfica, ou seja, visa satisfazer as necessidades de usuários que trabalham com mapas, sensoriamento remoto e observação da Terra e dispõe de uma lista de programas GIS selecionados para o Debian.
A equipe Debian-GIS também atua em conjunto para ajudar a manter o software GIS atual em derivados Debian, como UbuntuGIS , que fornece backports de pacotes GIS para Ubuntu, e OSGeo-Live para sua distribuição baseada no Ubuntu.
O Debian GIS se orgulha de que derivados estão lucrando com o trabalho dentro do Debian e está tentando estabelecer conexões fortes com esses derivados. Com o UbuntuGIS, a conexão é tão forte que um fluxo de trabalho comum foi criado, onde os desenvolvedores do UbuntuGIS estão injetando seu pacote diretamente no sistema de controle de versão do Debian GIS. Para ter certeza de que não haverá conflitos com as revisões do Debian, deve-se prestar atenção à numeração das revisões.

\subsection{Ubuntu-GIS}

Após adicionar o repositório UbuntuGIS correspondente à sua distribuição (em sources.list), o usuário pode instalar facilmente em sua máquina cada um dos aplicativos GIS, através do Synaptic Package Manager ou digitando sudo apt-get install nome do programa na linha de comando, como qualquer outro pacote. 
O Ubuntu-GIS basicamente reempacota os pacotes do Debian-GIS nas diferentes distribuições do Ubuntu. Uma vez que não é aconselhável instalar os pacotes diretamente pois os mesmos tem compatibilidade de fonte mas a compatibilidade binária não tem garantia de funcionamento, além até da possibilidade dos pacotes terem nomes diferentes devido a ambientes de compilação diferentes. Mas devido a compatibilidade pode ser que funcione corretamente com a realização de alguns ajustes.

\subsection{OSGeoLive}

É um DVD independente, pen drive USB e máquina virtual, baseado no Lubuntu. Inclui mais de 50 dos melhores aplicativos geoespaciais de código aberto, pré-configurados com dados, visões gerais do projeto e inicializações rápidas, traduzidos em vários idiomas. É uma excelente ferramenta para demonstrar GeoSpatial Open Source, usando em tutoriais e workshops, ou fornecendo a potenciais novos usuários. Ele permite que o usuário experimente uma ampla variedade de software geoespacial de código aberto sem instalar nada. É composto inteiramente de software livre, permitindo que seja livremente distribuído.Ele fornece aplicativos pré-configurados para uma variedade de casos de uso geoespacial, incluindo armazenamento, publicação, visualização, análise e manipulação de dados. Ele também contém conjuntos de dados de amostra e documentação.
Para experimentar os aplicativos, basta inserir o DVD ou pen drive USB no computador ou máquina virtual. Reinicializar o computador (verificar a ordem do dispositivo de inicialização, caso necessário), pressionar “Enter” para iniciar e fazer login. Selecionar e executar os aplicativos no menu “Geoespacial”.
OSGeoLive é um projeto da Fundação OSGeo, a Fundação OSGeo é uma organização sem fins lucrativos que apoia o desenvolvimento, promoção e educação de software de código-fonte aberto geoespacial.


\subsection{Como contribuir}

Do desenvolvedor ao usuário, há uma longa cadeia de tarefas nas quais existem a possibilidade de realizar algumas atividades. Primeiro, é preciso se manter informado sobre o panorama do software em GIS e / ou OpenStreetMap. Portanto, o usuário pode ajudar a monitorar o cenário GIS para softwares novos e atualizados e manter a equipe informada. O software a ser empacotado é escolhido de acordo com critérios como a necessidade do usuário e a consistência da distribuição.
Uma vez no Debian, o software é monitorado por sua qualidade e os bugs são corrigidos, se possível em colaboração com o (s) mantenedor (es) original (is). Portanto, o usuário pode ajudar fazendo a triagem de bugs, procurando por patches ou criando-os, depois testando as correções e informando a equipe. Todo esse trabalho não seria muito útil se permanecesse confidencial ou confinado aos pacotes fonte do Debian e suas distribuições derivadas. Também é dedicado algum tempo para anunciá-lo para o mundo via www.debian.org e para facilitar a integração de novos membros. Traduzir as descrições dos pacotes (se o usuário falar outro idioma além do inglês), marcar os pacotes com metadados e enviar capturas de tela também ajudará os usuários a encontrar o software que atende às suas necessidades.
Existem muitas maneiras de contribuir para o projeto e para isso é necessário entrar em contato com a equipe Debian-GIS em debian-gis@lists.debian.org.

\subsection{Tasks}

O Debian GIS Debian Pure Blend é organizado por tasks, que agrupam pacotes em torno de temas amplos como serviços web e estação de trabalho. As tasks listam programas que já estão empacotados no Debian assim como pacotes em preparação. Os arquivos de tasks não são hospedados nos repositórios Debian GIS, mas no repositório Debian Blends, em uma área de trabalho de computador normal, o usuário provavelmente desejará instalar pelo menos a task da estação de trabalho que contém os aplicativos GIS mais usados. 
Os meta-pacotes do Debian GIS Blend podem ser instalados em uma instalação normal do Debian. Como isso pode ser feito depende de qual versão do Debian você está executando.  
Estes são os metapacotes / tasks que podem ser instalados usando o comando apt-get install (nome do metapacote):
• gis-data - dados Debian GIS
• gis-gps - programas relacionados a GPS
• gis-osm - Programas relacionados ao OpenStreetMap
• gis-remotesensing - Sensoriamento remoto e observação da Terra
• gis-statistics - Estatísticas com dados geográficos
• gis-web - Apresentar informações geográficas via web mapserver
• gis-workstation - estação de trabalho de Sistemas de Informação Geográfica (GIS)
• gis-devel - pacotes de desenvolvimento GIS

\section{Sistemas Debian}

O sistema Debian ou Debian GNU/Linux é um sistema operacional que faz parte do projeto Debian, que é uma organização de pessoas com interesse em comum de criar um sistema operacional livre. Nesse projeto os membros são voluntários, e o sistema operacional Debian, que é um conjunto de programas básicos e utilitários que fazem o computador funcionar, pode ser adquirido gratuitamente na web. O sistema Debian é muito conhecido devido ao seu poderoso sistema de gerenciamento de pacotes (APT, Advanced Packaging Tool ou Ferramenta de Empacotamento Avançada) que permite a instalação de novos pacotes, além da atualização e remoção dos pacotes antigos de forma limpa e relativamente fácil. O Debian possui acesso a repositórios online com uma quantidade enorme de pacotes, sendo oficialmente apenas softwares livres. Softwares não-livres também podem ser baixados e instalados caso seja necessário.

\section{Requisitos para um software ser aceito no Debian-GIS}

No geral para um pacote ser aceito no Debian-GIS as regras são basicamente as mesmas para que o pacote seja aceito no Debian, pois é aconselhável seguir o caminho normal para o upload de pacotes Debian em pacotes Debian-Gis. Porém existem algumas edições no arquivo control do repositório Debian do software empacotado que devem ser realizadas e que são específicas para o Debian-Gis de acordo com a Política oficial Debian-GIS que pode ser encontrada através do link https://debian-gis-team.pages.debian.net/policy/policy.html.

\subsection{Mudanças no diretório Debian/control}

De acordo com a Plolítica Oficial do Debian-GIS, as seguintes mudanças devem ser feitas obrigatoriamente para que o pacote seja aceito:
Section (Seção): Deve ser “science” para o pacote de origem.
Priority (Prioridade): Deve ser “optional” a menos que proibido pela política Debian.
Maintainer (Mantenedor): O mantenedor deve ser Debian GIS Project <pkg-grass-devel@lists.alioth.debian.org>. O usuário pode se inscrever também nesta lista caso queira participar.
Uploaders: O usuário deve se increver como um uploader quando tiver um interesse significativo em um pacote.

\subsection{Busca por um sponsor}

Para quem não é um desenvolvedor oficial do Debian, para que o pacote seja posto no Debian é necessária a busca por um sponsor que revisará esse pacote e realizará a operação de upload. Existem vários Debian Developers na equipe Debian-GIS, infelizmente eles estão muito ocupados e podem não responder aos pedidos de patrocínio enviados para a Lista de Discussão dos Desenvolvedores Debian-GIS. Portanto, é recomendado seguir o processo normal no Debian e enviar um relatório de bug de Solicitação de Patrocínio (RFS). O site Debian Mentors é o melhor lugar para enviar seu pacote para atrair patrocinadores, o site também fornece um modelo para o seu relatório de bug RFS.

O relatório de bug RFS também deve ser copiado para a lista de discussão dos desenvolvedores GIS do Debian. A melhor maneira de fazer isso é usar o cabeçalho X-Debbugs-CC no relatório de bug. Adicione X-Debbugs-CC: pkg-grass-devel@lists.alioth.debian.org ao cabeçalho de e-mail da mensagem ou use a opção s no utilitário reportbug.

	%================================================================
\chapter{Pacote Debian do E-foto}
%================================================================
\section{Requisitos do E-foto}

Aqui será explicada a criação do pacote Debian do E-foto de acordo com as regras da Debian Policy \cite{bib:Ian}, cujo material de auxílio usado foi manual de empacotamento de Debian \cite{bib:Lucas}, o guia para novos mantenedores do Debian \cite{bib:Josip} e o guia rápido do empacotamento no Debian \cite{bib:Filho2020}.  

\subsection{Pacotes para a criação de um pacote de acordo com a política do Debian}

Primeiramente estão listados nos comandos abaixo os pacotes necessários para a criação de um pacote Debian de acordo com a política do Debian que deve ser criado num ambiente \textit{Debian unstable}. 

\begin{lstlisting}[language=bash]
	$ apt update
	$ apt install install autopkgtest blhc locales devscripts dh-make dput-ng git-buildpackage mc quilt spell tardiff tree
\end{lstlisting}

\begin{description}
	\item[Pacote autopkgtest] - enumera os testes do pacote e especifica suas dependências e requisitos.
	\item[Pacote blhc] - resolve possíveis problemas de hardening (processo de proteção) na compilação de pacotes que são escritos em C ou C++.
	\item[Pacote locales] - contém ferramentas para gerar definições de localidade de arquivos de origem, uma vez que quando um ambiente Debian unstable é criado configurações de localidade terão de ser feitas pois são reiniciadas.
	\item[Pacote devscripts] - contém diversos scripts que facilitam a vida do empacotador, pois seguem a política do Debian.
	\item[Pacote dh-make] - realiza a automatização de diversas etapas do empacotamento, além de gerar o diretório debian parcialmente preenchido de acordo com a política do Debian mais atual, evitando assim uma série de erros que podem ser cometidos no processo de empacotamento de algum software.
	\item[Pacote dput-ng] - é uma ferramenta de upload de pacotes para o Debian que fornece uma interface fácil de usar para instalações de hospedagem de arquivos de pacotes Debian. Ele permite que qualquer um que trabalhe com pacotes Debian carregue seu trabalho para um serviço remoto, incluindo ftp-master do Debian, mentors.debian.net, Launchpad ou outras facilidades de hospedagem de pacotes para mantenedores de pacotes Debian.
	\item[Pacote git-buildpackage] - ferramenta que facilita o manuseio de pacotes Debian em repositórios Git, uma vez que o git é uma forma de controle de versão tanto do software fornecido pelo upstream tanto como do próprio processo de empacotamento.
	\item[Pacote mc] - é um gerenciador de arquivos de tela inteira em modo texto. Ele usa uma interface de dois painéis e um subshell para execução de comandos. Inclui um editor interno com destaque de sintaxe e um visualizador interno com suporte para arquivos binários.
	\item[Pacote quilt] - é um gerenciador de patches, acompanhando as mudanças que cada um deles faz. E funciona os organizando logicamente como uma pilha e você pode aplicá-los, desaplicá-los e atualizá-los facilmente viajando para a pilha. Esse pacote só é necessário caso exista algum patch de mudança no empacotamento.
	\item[Pacote spell] - é um programa de verificação ortográfica que imprime cada palavra incorreta em uma linha própria, muito bom para garantir que nenhum arquivo do pacote conterá erros ortográficos, é de muita importância que o empacotador utilize o mesmo após a finalização de cada arquivo para realizar a verificação, pois encontrará dificuldades de ter seu pacote aceito pelo Debian caso tenha erros de ortografia.
	\item[Pacote tardiff] - compara o conteúdo de dois tarballs e relata quaisquer diferenças encontradas entre eles. Seu uso é principalmente para gerentes de lançamento que podem usá-lo como uma ferramenta de controle de qualidade para garantir que nenhum arquivo tenha sido deixado de lado acidentalmente ou adicionado por engano. TarDiff suporta tarballs compactados, estatísticas de diferenças e supressão de mudanças GNU autotool. Esse pacote é muito usado pois como pode ser visto no manual do desenvolvedor do Debian, o código fonte do upstream geralmente vem no formato .tar.gz.
	\item[Pacote tree] - é um comando recursivo de listagem de diretórios que produz uma listagem de arquivos em formato de árvore. Muito usado para realizar a verificação de como ficou o empacotamento, se cada arquivo e diretório estão localizados em seu devido lugar.
	\item[Pacote lintian] - é um sistema que disseca os pacotes Debian e relata bugs e violações de políticas. Ele contém verificações automatizadas para muitos aspectos da política do Debian, bem como algumas verificações para erros comuns. Ele usa um diretório de arquivo, denominado “laboratório”, no qual armazena informações sobre os pacotes que examina. Ele pode manter essas informações entre várias chamadas para evitar a repetição de operações caras de coleta de dados. Isso torna possível verificar o repositório Debian completo em busca de bugs, em um tempo razoável. Este pacote é útil para todas as pessoas que desejam verificar os pacotes Debian quanto à conformidade com a política Debian. Cada mantenedor do Debian deve verificar os pacotes com esta ferramenta antes de enviá-los.
\end{description}
 
\section{Criação do Pacote Debian do E-foto}
%mostrar um passo a passo

\subsection{Verificação de Intenção de Empacotar}

Antes de realizar o empacotamento de qualquer software para o Debian deve ser feita a verificação se não existe esse mesmo software já empacotado ou se algum outro programador já declarou a intenção de empacotá-lo. Para isso é necessário acessar o site \url{https://wnpp.debian.net/} ou o site \url{https://bugs.debian.org/cgi-bin/pkgreport.cgi?pkg=wnpp;dist=unstable} e procurar o nome do software que será empacotado, pois pacotes duplicados não são aceitos pelo Debian. Ao fazer isso com o E-foto foi encontrado o bug de intenção de empacotar \url{https://bugs.debian.org/cgi-bin/bugreport.cgi?bug=899283}, mas essa tentativa não obteve sucesso pois na época o E-foto ainda estava em Qt 4. Agora com o empacotamento da sua versão em Qt 5 deverá ser criado um outro bug para que o novo pacote seja aceito.

\subsection{Processo de criação de uma jaula de ambiente Debian unstable}

O primeiro passo para a criação de um pacote de acordo com a política do Debian é a obtenção de uma ambiente \textit{Debian unstable}, já que todo empacotamento deve ser feito em um ambiente \textit{Debian unstable} (com exceções de modificações ou correções em pacotes já publicados no \textit{Debian stable}), pois o pacote vai entrar no \textit{Debian unstable}, depois passar para \textit{experimental} até poder ser classificado e utilizado no \textit{stable}, mas para a realização desse empacotamento o empacotador não precisará usar uma versão \textit{unstable} do Debian caso não seja sua preferência, será suficiente a criação de apenas um ambiente chamado também de \textit{jaula-sid} que é justamente um diretório dentro do \textit{Debian stable} em que o desenvolvedor estará confinado e que dentro dele estará o \textit{Debian unstable} e essa jaula pode ser criada a partir dos comandos: 

\begin{lstlisting}[language=bash]
	$ apt update
	$ mkdir /jaula-sid
	$ debootstrap sid /jaula-sid/ftp.br.debian.org/debian
	$ chroot /jaula-sid/
	$ echo proc /proc proc defaults 0 0 >> etc/ fstab
	$ wget http://bit.ly/bash-rc-txt
	$ apt install autopkgtest blhc locales devscripts dh-make dput-ng git-buildpackage mc quilt spell tardiff tree lintian
	$ apt-get clean
	$ dpkg-configure locales tzdata
\end{lstlisting}

\begin{description}

	\item[Comando mkdir /jaula-sid] - cria um diretório na raiz com o nome de jaula-sid. O nome \textbf{sid} vem do nome do Debian \textit{unstable}.
	\item[Comando debootstrap sid /jaula-sid/\textit{ftp.br.debian.org/debian}] - realiza o download e a instalação de um ambiente Debian unstable no diretório jaula-sid a partir do link indicado para download da imagem.
	\item[Comando chroot /jaula-sid/] - altera a raíz para o diretório jaula-sid tornando-o isolado ou enjaulado de forma em que ele não conseguirá acessar nada fora dele mesmo. Com isso, o empacotador ficará a partir daqui utilizando só a versão \textit{unstable} do Debian. É desse método que foi inspirado o nome jaula para o diretório de empacotamento.
	\item[Comando echo proc /proc proc defaults 0 0 >> etc/ fstab] - cria um dispositivo proc (é um diretório especial onde ficam todas as informações de depuração do \textit{kernel}, também se encontram algumas configurações que habilitam e desabilitam o suporte à alguma coisa no \textit{kernel}) que vai garantir o funcionamento correto da jaula.
	\item[Comando wget \textit{http://bit.ly/bash-rc-txt}] - realiza o download de um arquivo de texto que contém uma série de comandos de configuração e ele deve ser colocado no \textit{/etc/bash.bashrc} pois esse diretório é lido sempre que ocorre a troca de ambiente para o Debian unstable.
	\item[Comando cat bash-rc-txt >> /etc/bash.bashrc] - coloca o arquivo no diretório correto como explicado acima.

\end{description}

O conteúdo do bash-rc-txt é o seguinte:
\begin{verbatim}
alias ls='ls --color=auto' 
alias tree='tree -aC' 
alias debuildsa='dpkg-buildpackage -sa -ksua\_chave\_gpg' 
alias uscan-check='uscan --verbose --report' 
alias debcheckout='debcheckout -a' 
export DEBFULLNAME="seu\_nome\_completo\_sem\_acentos/cedilha" 
export DEBEMAIL="seu\_e-mail" export EDITOR=mcedit 
export LANG=C.UTF-8 export LANGUAGE=C.UTF-8 
export LC\_ALL=C.UTF-8 
export QUILT\_PATCHES=debian/patches 
export PS1='JAULA-SID-\u@\\h:\\w\$ ' 
mount /proc 
\end{verbatim}

Agora é o momento de realizar a instalação de alguns pacotes essenciais para o funcionamento correto da jaula, é importante frisar que a jaula deve ser mantida o mais limpa possível, ou seja, só pacotes realmente essenciais devem ser instalados pois isso pode afetar na hora da construção e compilação do pacote. Esse procedimento é o descrito na subseção acima. O próximo passo de configuração do ambiente de trabalho é setar o \textit{locales} (que define que idiomas serão usados e qual o idioma padrão do ambiente de trabalho, geralmente pt/\_BR.UTF-8) e o fuso horário (São paulo, cidade que está no centro do fuso) através do comando \textit{dpkg-configure locales tzdata}. Agora deve ser feita a configuração do \textit{lintian},e para realizar essa configuração deve ser criado o arquivo \textit{/root/.lintianrc} e colocar no mesmo as seguintes linhas: 

\begin{verbatim}
display-info = yes 
pedantic = yes 
display-experimental = yes 
color = auto.
\end{verbatim}

E usar o comando \textit{apt-get clean} para eliminar arquivos desnecessários e poupar espaço para iniciar o empacotamento.

Agora a próxima etapa deve ser assinar o pacote com a chave GPG do empacotador, caso seja da escolha dele, da seguinte maneira: fora da jaula, exporte as suas chaves GPG (privada e pública) com os comandos:

\begin{lstlisting}[language=bash]
	$ gpg -a --export nr_da_chave > nr_da_chave.pub 
	$ gpg -a --export-secret-keys nr\_da\_chave > nr\_da\_chave.key
\end{lstlisting} 

Mova os arquivos para dentro da jaula \textit{unstable}, volte para a jaula com o comando \textit{chroot /jaula-sid/}, importe as chaves e remova os arquivos. Para importar use os seguintes comandos: 

\begin{lstlisting}[language=bash]
	$ gpg --list-keys 
	$ echo "pinentry-mode loopback" >> ~/.gnupg/gpg.conf 
	$ gpg --import chave.key chave.pub 
\end{lstlisting}

Agora para habilitar a chave para assinatura de pacotes o empacotador deve editar o arquivo \textit{/etc/devscripts.conf} e inserir o número completo da chave GPG na linha \textit{DEBSIGN\_KEYID}, descomentando-a. Com isso, finalmente a jaula de trabalho está pronta e configurada para começar o empacotamento.

\subsection{Começando o processo de empacotamento}

%nomes de pacote só podem conter letras minúsculas, números e símbolos sendo que no mínimo dois algarismos.
Primeiramente o empacotador deve instalar o programa e testá-lo para ver se ele realmente funciona e se vale a pena empacotá-lo.

Após feito isso, para começar realmente o empacotamento, é preferível a criação dentro da jaula de trabalho um diretório com o nome genérico do pacote (por exemplo \textit{pkgnome-do-programa}) através do comando:

\begin{lstlisting}[language=bash]
	$ mkdir pkgnome-do-programa 
\end{lstlisting} 

Dentro desse diretório a primeira etapa é obter o código fonte do \textit{upstream} que de acordo com o manual de desenvolvedores deve estar preferencialmente no formato \textit{.tar.gz} (tarball) e descompactá-lo. A política geral dos softwares do \textit{Debian} prevê que para realizar o empacotamento, o diretório raiz deve ser nomeado da seguinte forma: \textbf{nomedoprograma-numerodeversão}. Logo após isso, o empacotador deve preferencialmente verificar o licenciamento do programa e isso pode ser feito de duas formas distintas, uma é através do comando: \textit{lincensecheck} * (que dará como resultado a licença de cada arquivo dentro do código fonte fornecido pelo upstream) ou através do comando\textit{ egrep -sriA25 '(public dom|copyright)' | less} (que toda vez em que as expressões \textit{public dom} ou \textit{copyright} forem citadas nos arquivos-fontes as 25 linhas posteriores serão mostradas, assim mostrando o que o autor do código explicitou sobre a licença). Esse procedimento deve ser feito com calma e de maneira bem minuciosa para novos pacotes pois o Debian não aceita nenhum mínimo detalhe que comprometa o fato de o código ser livre.

O Próximo passo será um dos mais importantes pois é a "debianização" do código fonte (a criação do diretório Debian dentro do código junto com seus arquivos necessários de acordo com a política do Debian) e isso é feito através do comando:

\begin{lstlisting}[language=bash]
	$ dh_make -f ../.tar.gz -c <licença> 
\end{lstlisting} 

\subsection{Modificando o conteúdo do diretório Debian do pacote}

Os primeiros arquivos do diretório Debian a ser modificados são os que são obrigatórios para o empacotamento:\textbf{ Changelog, Control, Copyright e Rules.} Devido a utilização do pacote \textit{dh-make}, esses arquivos já virão previamente preenchidos com algumas informações que deverão ser modificadas de acordo com o conteúdo do software que está sendo empacotado.

Começando pela edição do \textit{debian/changelog} que no caso de ser a primeira vez em que o software está sendo empacotado, é só alterar o modelo para se adequar ao seu pacote com informações básicas (alterar de \textit{unstable} para \textit{experimental} em caso de ser o primeiro empacotamento do programa, por exemplo), mas futuramente toda modificação no pacote que vai gerar uma revisão pelo Debian deve ser documentada nesse arquivo. 

O segundo a ser editado será o \textit{debian/control} que também já vem com o formato preenchido sendo necessário só alterar os valores de acordo com o programa, e esses valores são divididos em dois parágrafos, o superior trata do código fonte e o inferior trata do binário que será gerado.

Sobre os campos do fonte (parágrafo superior):
\begin{verbatim}
Source: nome do programa.
Section: é o nome da seção em que se enquadra a funcionalidade do programa que
será empacotado, essas funções e suas descrições podem ser vistas no site 
https://packages.debian.org/unstable/.
Priority: prioridade em que o programa deve ser instalado no Debian, serve
para controlar quais pacotes são instalados junto com o Debian em determinadas 
configurações e essa prioridade deve ser determinada apenas pela funcionalidade
que o programa fornece ao usuário.
Maintainer: nome do mantenedor do pacote, geralmente junto com e-mail.
Build-depends: são os pacotes que tem que ser obrigatoriamente instalados para
que instalação do programa, como os pacotes são para desenvolvimento, geralmente
eles têm -dev no final. O pacote que necessariamente o empacotador está usando
é o debhelper-compat (=13).
Standards-version: A versão mais recente dos padrões (o manual de política e
textos associados) com os quais o pacote está em conformidade. 
Homepage: É a página da internet onde o código fonte pode ser obtido.
\end{verbatim}

Sobre o binário (parágrafo inferior):
\begin{verbatim}
Package: nome do pacote que será gerado.
Architecture: aqui deve ser especificado pra qual tipo de arquitetura o 
empacotador quer que seu pacote seja gerado. Geralmente são any 
(programas em linguagem compilada) ou all (programas em linguagem 
interpretada), mas caso o programa só funcione em uma arquitetura específica,
ela deve ser especificada aqui nesse campo.
Depends: são auxiliares que na construção do binário busca dependências
do código que não foram especificadas. 
Description: é dividida em duas partes, a descrição curta e a descrição longa.
A descrição curta deve ser rápida e objetiva com as funcionalidades do programa.
Já a longa geralmente é copiada da descrição do autor que é dada na homepage do
upstream porém deve ser formatada de acordo com a politica Debian, toda linha 
tem que ser indentada com um espaço em branco, o comprimento de cada linha é de
no máximo 80, caso exista uma linha em branco entre parágrafos ela deve ser
preenchida com um ponto final e geralmente é preferível que esse texto seja
escrito de maneira impessoal.
\end{verbatim}

Existem outros campos possíveis mas esses listados são os obrigatórios e necessários, esses outros campos podem ser vistos nesse site \url{https://www.debian.org/doc/debian-policy/ch-controlfields.html}

O próximo a ser preenchido é o \textit{debian/copyright} com os seguintes campos:
\begin{verbatim}
Format: já vem preenchido com o formato que está sendo seguido para a realização
do  copyright.
Upstream-Name: Nome do programa.
Upstream-Contact: Geralmente é colocado aqui o endereço da página do upstream
usada para o envio de bugs.
Source: Homepage do usptream.
Cada arquivo que tiver uma licença diferente deve ser especificado com esses
3 campos:
Files: nome do arquivo.
Copyright: autor e data.
License: nome da licença.
Geralmente todo código tem uma especificação de lincença  para o código fonte
e outra para o diretório debian no caso de o autor do programa e o autor do
empacotamento sejam diferentes. Ou pode ser colocados o mesmo para que os 
códigos possam ser misturados.
License: Explicação completa feita sobre o que diz a licença. 
\end{verbatim}

Agora será editado o campo \textit{debian/rules}. Este campo já vem comentado com algumas sugestões de preenchimento. Devem ser deixadas as seguintes linhas:
\begin{verbatim}
#!/usr/bin/make -f
#export DH\_VERBOSE = 1
#export DEB\_BUILD\_MAINT\_OPTIONS = hardening=+all
#export DEB\_CFLAGS\_MAINT\_APPEND = -Wall -pedantic
#export DEB\_LDFLAGS\_MAINT\_APPEND = -Wl, --as-needed

%:
dh $@

\end{verbatim}
O restante deve ser apagado e cada linha de comentário deixada só pode ser descomentada em caso de necessidades especiais do programa que serão sanadas com as mesmas. O arquivo \textit{rules} é o \textit{makefile} dentro do diretório Debian do pacote e é nele que devem ser especificados quaisquer peculiaridades da construção do programa, para que o pacote possa ser montado de maneira bem-feita. Ao final do empacotamento, após a realização dos testes do pacote, essa linhas de comentários que não tiveram utilidade devem ser apagadas.

Por último será preenchido o \textit{debian/watch} com as seguintes informações:
\begin{verbatim}
Version: a versão do watch que está sendo usada (deve ser usada sempre
a mais atual).
As próximas linhas devem ser preenchidas com links onde devem ser 
procuradas possíveis atualizações feitas pelo upstream no código fonte.
Geralmente são preenchidos com links de sites usados para o controle de 
versão de programas ou códigos como o Github ou Sourceforge.
É importante deixar claro que pra realizar o empacotamento o empacotador
deve apagar todo lixo, comentário e coisas desnecessárias. O pacote deve
ser o mais limpo e leve possível.
\end{verbatim}

\subsection{Construção do pacote}

Após todas as etapas citadas anteriormente, é hora de usar o comando \textit{debuild} no diretório raiz e esperar a compilação do programa e criação do pacote. Quando acabar muito possivelmente aparecerão alguns erros \textit{lintian} que deverão ser corrigidos para que o pacote esteja de acordo com a política do Debian, a maneira mais fácil de corrigir esses erros \textit{lintians} é acessando o site \url{https://lintian.debian.org/tags.html} e procurar a explicação do erro que está acontecendo e como resolvê-lo. Quando o seu pacote estiver sem erros \textit{lintians}, ou só com \textit{lintians} que não são solucionáveis (não podendo ser \textit{lintians} de erro ou algo relacionado a obrigações do pacote ou empacotador) chegou a hora de realizar as checagens finais e começar a buscar as maneiras de subir esse pacote para o Debian.

\subsection{Testes finais do pacote}

Logo após a construção do pacote e resolução dos erros \textit{lintians}, o empacotador deve realizar mais alguns testes antes de enviar seu pacote para o Debian, primeiramente ele deve passar o comando \textit{spell} em todos os arquivos que ele mesmo criou do diretório Debian, depois ele deve usar o comando \textit{tree nomedopacote} dentro do diretório Debian do pacote para verificar como foi feita a construção do mesmo, ou seja, se cada arquivo foi para o lugar em que deveria após o empacotamento. E nessa verificação é importante perceber se o \textit{dh\_compress} fez seu papel corretamente, já que ele é uma ferramenta do \textit{debhelper} que comprime todo arquivo com tamanho maior que 4k (com a exceção de arquivos \textit{copyright}, \textit{.html} e outros arquivos web, arquivos de imagens, e arquivos que aparentam já estarem comprimidos com base nas suas extensões), essa compressão de arquivos é feita de acordo com a política do Debian e o \textit{dh\_compress} é chamado automaticamente junto com o comando \textit{debuild} usado para a construção do pacote.

Após essa etapa o empacotador deve verificar novamente todos os arquivos do diretório Debian e usar o comando \textit{debuild} duas vezes para ter certeza de que não ocorrerá nenhum novo erro na construção do pacote. A ultima etapa é usar o \textit{cowbuilder} para testar a construção do pacote em uma jaula \textit{unstable} limpa e isso é feito através dos seguintes comandos:

\begin{lstlisting}[language=bash]
	$ cowbuilder create
	$ cowbuilder update
	$cowbuilder build arquivo.dsc
\end{lstlisting} 

Esses comandos vão criar o ambiente \textit{Debian unstable} limpo, atualizar o mesmo e testar a construção do pacote nesse ambiente respectivamente. Esse teste é de total importância pois durante o processo de empacotamento pode ser gerada uma poluição na jaula que acabará influenciando no resultado final, e com esse teste em um ambiente limpo ocorre uma simulação de como o pacote será construído e testado nos servidores do Debian quando for feito seu \textit{upload}.
	%================================================================
\chapter{Resultados}
%================================================================

\section{Testes}

\subsection{Ubuntu}
%testar a instalação em sistemas ubuntu 18.04, e 20.04
Para começar os testes no \textit{Ubuntu}, foi usada a versão 20.04 do mesmo onde primeiramente foi testado o pacote gerado através do empacotamento feito na \textit{Jaula-SID} que, como explicado anteriormente, cria um ambiente \textit{Debian Unstable} dentro do próprio \textit{Ubuntu}, gerando assim um pacote que contém dependências mais atualizadas do que as disponíveis no \textit{Ubuntu}. Este empacotamento segue as politicas do \textit{Debian} e também se baseia parcialmente no \textbf{Manual de Empacotamento de Debian} que pode ser obtido em sua versão traduzida para o português através do endereço:

\textit{https://www.debian.org/doc/manuals/packaging-tutorial/packaging-tutorial.pt.pdf}.
 
Devido a esse problema de compatibilidade os primeiro testes acabaram gerando erro durante a instalação do pacote, e esses erros não eram contornáveis, uma vez que esses pacotes cujo o pacote E-foto tinha como dependência não tinham essas versões disponíveis de maneira alguma no \textit{Ubuntu 20.04}. 

\textit{Inserir imagens dos erros ocorridos durante o teste}.

Após algum tempo de estudo e pesquisas foi decidido o teste através da realização de um empacotamento diretamente no Ubuntu 20.04, sem a utilização da \textit{Jaula-SID} ou qualquer outro tipo de ambiente \textit{Debian Unstable}. Seguindo novamente de maneira parcial o \textbf{Manual de Empacotamento de Debian} e também o passo-a-passo que foi descrito aqui anteriormente, ocorreu a criação de um novo pacote com sucesso, porém durante a compilação do mesmo um novo erro \textit{lintian} apareceu, que trazia a mensagem retratando que o sistema operacional em que um novo pacote deve ser criado é o \textit{Debian Unstable} e que a versão do sistema operacional usado estava entrando em conflito com essa política. Mas mesmo com essa nova mensagem de advertência, a compilação ocorreu com sucesso assim como a geração do pacote e após isso, foi só instalar o pacote através do comando \textit{sudo dpkg -i nome-do-pacote}, buscar o caminho do diretório onde o \textit{E-foto} foi instalado através do terminal e utilizar o comando \textit{./efoto} no diretório onde fica o executável, geralmente \textit{usr/bin}.

\textit{Inserir imagens do resultado final do teste}.


\subsection{Debian Stable}
%testar a instalação em sistemas debian puro

\subsection{Debian Unstable}
%testar a instalação em sistemas debian puro

Para começar o teste do funcionamento do pacote do software E-foto no sistema operacional \textit{Debian} em sua versão instável completa, que é o ambiente correto para a realização destes teste devido á ser neste ambiente que o próprio \textit{Debian} realizará a primeira etapa de testes quando o pacote for finalizado e disponibilizado, é necessária a obtenção dessa versão instável do \textit{Debian}, e para isso são necessários alguns passos, uma vez que essa versão não é disponibilizada diretamente para download no website do \textit{Debian}. O método escolhido para instalação da versão instável do \textit{Debian} foi através da alteração do arquivo \textbf{source.list} através do terminal de comando na versão do \textit{Debian testing} mais atual. 

O primeiro passo é a realização do download da versão mais atual do \textit{Debian testing} mais atual, que pode ser obtida através do endereço \textit{}https://www.debian.org/devel/debian-installer/ e que deverá ser instalada em uma máquina própria ou em ma máquina virtual. Após a instalação do sistema operacional Debian Testing, a próxima etapa é a edição do arquivo \textbf{source.list} através da linha de comando no caminho \textit{/etc/apt/source.list} substituindo o seu conteúdo com pelo seguinte:
 \begin{verbatim}
deb http://ftp.br.debian.org/debian/ unstable main contrib non-free	
deb-src http://ftp.br.debian.org/debian/ unstable main contrib non-free
\end{verbatim}
Depois é preciso salvar e sair. Novamente no terminal de comando deve ser usado o comando \textit{sudo apt update} para que todos as listas de pacotes possam ser atualizadas, obtendo assim as versões mais recentes de cada pacote disponível no local especificado pelo \textbf{source.list} e depois o ultimo passo é o \textit{sudo apt upgrade} que instalará as versões mais recentes de todos os pacotes que já estão instalados no sistema baseados no local especificado pelo arquivo \textbf{source.list}. Com o fim da realização destes passos, a versão instável do \textit{Debian} estará disponível para uso e consequentemente para a realização do teste do pacote.

Para testar o funcionamento do pacote, é preciso ir através do terminal de comando até o local onde está o arquivo .deb do E-foto e utilizar o comando \textit{sudo dpkg -i nome-do-pacote} que a instalação ocorrerá automaticamente, porém para que isso aconteça o pacote \textbf{libgdal.dev} deve estar instalado previamente, após a instalação ser concluída, basta ir até o local onde foi instalado o E-foto, buscar o seu arquivo executável, que geralmente é instalado no \textit{usr/bin}, e usar o comando \textit{./efoto} que o programa será executado perfeitamente.

\textit{Inserir imagens da instalação com sucesso}.

\textit{Inserir imagens do resultado final do teste}.

\subsection{Debian-GIS}
%testar a instalação em sistemas debian gis
	%================================================================
\chapter{Conclusão}

OI
%================================================================

Texto da conclusão.
	
	
	%\index{Introdução!Capítulo}.
	
	%XXXXXXXXXXXXXXXXXXXXXXXXXXXXXXXXXXXXXXXXXXXXXXXXXXXXXXXXXXXXXXXX
	% ELEMENTOS POS-TEXTUAIS
	%XXXXXXXXXXXXXXXXXXXXXXXXXXXXXXXXXXXXXXXXXXXXXXXXXXXXXXXXXXXXXXXX
	\backmatter % inicia a área dos elementos pós-textuais
	%XXXXXXXXXXXXXXXXXXXXXXXXXXXXXXXXXXXXXXXXXXXXXXXXXXXXXXXXXXXXXXXX
	
	%===========================================================
	% Referencias via BibTeX
	%===========================================================
	
	\citeoption{abnt-options4}
	\bibliography{abnt-options4,bibliografia}
	
	%===========================================================
	\postextualchapter*{Glossário} % elemento opcional
	%===========================================================
	
	\definicao{termo 1}{significado}
	\definicao{termo 2}{significado}
	\definicao{termo 3}{significado}
	
	%XXXXXXXXXXXXXXXXXXXXXXXXXXXXXXXXXXXXXXXXXXXXXXXXXXXXXXXXXXX
	% Apêndices (opcionais)
	%XXXXXXXXXXXXXXXXXXXXXXXXXXXXXXXXXXXXXXXXXXXXXXXXXXXXXXXXXXX
	\appendix % inicia os apêndices
	%XXXXXXXXXXXXXXXXXXXXXXXXXXXXXXXXXXXXXXXXXXXXXXXXXXXXXXXXXXX
	
	%===========================================================
	
	%===========================================================
	\include{capAntigoProjeto}
	\include{capProfile}
	\include{capFicha}
	

	%/\/\/\/\/\/\/\/\/\/\/\/\/\/\/\/\/\/\/\/\/\/\/\/\/\/\/\/\/\/
\end{document}
%/\/\/\/\/\/\/\/\/\/\/\/\/\/\/\/\/\/\/\/\/\/\/\/\/\/\/\/\/\/
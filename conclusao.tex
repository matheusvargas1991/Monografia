%================================================================
\chapter{Conclusão}

O trabalho realizado permitiu que fosse percebida a grande diferença entre as formas de instalação do software E-foto em diferentes sistemas operacionais evidenciando assim os benefícios trazidos a partir do empacotamento do mesmo. Porém ainda assim foi deixado explicado em forma de tutorias de linguagem bem simples as maneiras corretas de se obter e instalar o E-foto nos sistemas operacionais mais usados.

Foi visto claramente que obter o pacote do E-foto simplifica bastante sua instalação se comparada com as outras formas de instalação mostradas e torná-lo um pacote aceito pelo \textit{Debian-GIS} facilitará a propagação dos benefícios do software e de suas funcionalidades para o nicho correto que se interessará por elas. Aumentando assim o grupo que se interessará pela utilização do E-foto. 

Outro ponto benéfico da realização do trabalho foi a expansão do conhecimento para além das salas de aulas e das disciplinas do curso de computação, já que foi necessário um certo aprendizado sobre o tema da cartografia uma vez que é sobre esse tema que o E-foto trata.

Por último foi constatada a importância do trabalho que vem sendo realizado pelo grupo que mantém o E-foto, pois gera um oportunidade gigantesca de aprendizado de alunos de áreas distintas, bem como o trabalho em equipe de aprimoramento do software e o acesso gratuito a uma estação fotogramétrica digital.     

Fazer parte do projeto E-foto é uma rara oportunidade de aprender fotogrametria e programação na prática, seja na sua criação, nos testes feitos ou na própria utilização, já que oportunidades como essa são difíceis durante a graduação. E com o avanço tecnológico que aumentou a importância da fotogrametria digital, ter um software livre e gratuito disponível para isso faz do E-foto um projeto diferenciado e muito importante. 
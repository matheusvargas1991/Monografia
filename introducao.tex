%================================================================
\chapter*{Introdução}
%================================================================

O software E-foto está em desenvolvimento desde 2004 pelo laboratório de fotogrametria da Faculdade de Engenharia da Universidade do Estado do Rio de Janeiro, e tem como foco oferecer uma \textbf{Estação Fotogramétrica Digital} gratuita e livre, que poderia ajudar os alunos a entender os princípios por trás da fotogrametria, evitando assim custos elevados que teriam impedido muitos de aprender sobre o assunto. O usuário interessado tem a possibilidade de instalá-lo tanto em sistemas operacionais \textit{Linux} (preferencialmente \textit{Ubuntu}) quanto \textit{Windows}, porém, mesmo após o lançamento de várias versões do programa, o código nunca foi empacotado de maneira que ele pudesse ser aceito como pacote oficial nos repositórios \textit{Debian}, uma vez que para isso é necessário seguir uma série de normas que são explicadas na política de empacotamento do \textit{Debian}, e que envolvem desde licenças do código que garantem o fato do mesmo ser livre, até regras sobre o próprio empacotamento do software.

\section*{Objetivo}

Esse projeto tem como objetivo adaptar o código do software E-foto e gerar um pacote  de maneira que ele possa ser aceito oficialmente nos repositórios \textit{Debian} e a partir disso possa participar do \textit{Debian-gis}, que é uma parte do \textit{Debian} direcionada para os usuários de \textbf{Sistemas de Informação Geográfica}. Com isso o software terá sua instalação facilitada, pois pacotes do \textit{Debian} são instalados com comandos simples e também a sua divulgação aumentada, pois estará num ambiente direcionado para pessoas com interesse semelhante às suas funcionalidades. 

Além desse objetivo principal, também será apresentado no projeto informações como tutoriais de instalação do software E-foto no \textit{Ubuntu} e no \textit{Windows}, assim como um explicação sobre o uso de suas principais aplicações, e também um tutorial de como é realizado o empacotamento de maneira correta de acordo com a política do Debian e seus documentos oficiais.

\section*{Estrutura da Monografia}

Esta monografia está dividida em seis capítulos: \textbf{Introdução, efoto, Pacote debian efoto, Debian-gis, Resultados e Conclusão}.

No primeiro e atual capítulo é fornecida uma explicação sobre o projeto do Efoto, uma explicação sobre o \textit{Debian} e sobre o \textit{Debian-gis}, assim como a motivação para este projeto.

No capítulo \textbf{efoto} está disponível a explicação completa sobre como instalar o software em suas diferentes formas, a partir de um tutorial testado em sistemas \textit{Ubuntu} e em \textit{Windows}, desde a obtenção do código fonte até a explicação sobre suas funcionalidades dentro do programa.

No capítulo \textbf{Pacote debian efoto} é apresentado um passo a passo sobre como realizar o empacotamento do efoto de acordo com a política do \textit{Debian}, além  dos passos essenciais para tal. Esse capítulo tem como objetivo seguir o  que é um documento oficial com explicações sobre o empacotamento.

No capítulo \textbf{Debian-gis} é explicado o que é o \textit{Debian}, o que é o \textit{Debian-gis} e o que é necessário para que o pacote do Efoto seja aceito no \textit{Debian-gis} depois de empacotado corretamente.

No capítulo \textbf{Resultados} são apresentados os resultados dos testes realizados de instalação e funcionamento do pacote efoto em diferentes sistemas operacionais.

O sexto e último capítulo apresenta um resumo sobre o foi desenvolvido, ideias para seu uso e aplicações além de sugestões de melhorias futuras.

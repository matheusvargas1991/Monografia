%================================================================
\chapter*{Introdução}
%================================================================

\begin{epigrafeonline}
\hfill Texto da epígrafe \textit{in locu}.\\
\hspace*{\fill}\textit{Autor}\\
\end{epigrafeonline}
% explique aqui a importância de seu projeto
O software E-foto está em desenvolvimento desde 2004 pelo laboratório de fotogrametria da Escola de Engenharia da Universidade Estadual do Rio de Janeiro e tem como foco oferecer um conjunto simples de software (uma Estação Fotogramétrica Digital gratuita) que poderia ajudar os alunos a entender os princípios por trás da fotogrametria, evitando assim custos que teriam impedido muitos de aprender sobre fotogrametria. Esse objetivo foi alcançado por meio do desenvolvimento de um software fotogramétrico de fácil utilização e gratuito. Porém após o lançamento de várias versões do programa o usuário interessado no mesmo tem a oportunidade de instalar e utilizar no ubuntu e no windows, mas o código nunca foi empacotado de maneira em que ele pudesse ser upado e aceito no Debian, uma vez que para isso é necessário seguir uma série de normas que são explicadas na política do Debian, e que envolvem desde licenças do código que garantem o fato do mesmo ser livre até regras sobre o próprio empacotamento do software.

\section*{Objetivo}

Esse projeto tem como objetivo adaptar o código do software E-foto e gerar o pacote Debian de maneira que ele possa ser aceito oficialmente pelo Debian e a partir disso possa participar do Debian-gis que é uma parte do Debian direcionada para os usuários interessados em um Sistema de Informação Geográfica. Com isso o software terá sua instalação facilitada pois pacotes do Debian são instalados facilmente e também a sua divulgação aumentada pois estará num ambiente direcionado para pessoas com interesse semelhante as suas funcionalidades. Além desse objetivo principal, também será apresentado no projeto informações como tutoriais de instalação do software E-foto no Ubuntu e no Windows, assim como um explicação sobre o uso de suas principais aplicações, e também um tutorial de como é realizado o empacotamento de maneira correta de acordo com a política do Debian e seus documentos oficiais, sendo de suma importância pois é a primeira vez que tais informações são documentadas e ficarão disponíveis para usuários em potencial.

\section*{Estrutura da Monografia}

Esta monografia está dividida em seis capítulos, os quais são: Introdução, efoto, pacote debian efoto, debian-gis, resultados e Conclusão.

No primeiro e atual capítulo foi fornecida uma explicação sobre o projeto do Efoto, uma explicação sobre o Debian e sobre o Debian-gis, assim como a motivação para este projeto.

No capítulo efoto é onde está disponível a explicação completa sobre como instalar o software Efoto em suas diferentes formas, a partir de um tutorial testado em ubuntu e em windows, desde a obtenção do código fonte até a explicação sobre suas funcionalidades dentro do programa.

No capítulo pacote debian efoto está explicado um passo a passo sobre como realizar o empacotamento do software do efoto de acordo com a política do Debian, além da explicação dos passos essenciais para tal. Esse capítulo tem como objetivo seguir o guia dos novos mantenedores do Debian, que é um documento oficial com explicações sobre o empacotamento.

No capítulo debian-gis é explicado o que é o Debian, o que é o Debian-gis e o que é necessário para que o pacote do Efoto seja aceito no Debian-gis depois de empacotado corretamente.

No capítulo resultados são apresentados os resultados dos testes realizados de instalação e funcionamento do pacote efoto em diferentes sistemas operacionais.

O sexto e último capítulo apresenta um resumo sobre o foi desenvolvido, ideias para seu uso e aplicações além de sugestões de melhorias futuras.

 
% ---------------------------------------------------------------
% ---------------------------------------------------------------
% Modelo de Trabalho Acadêmico utilizando classe repUERJ para
% elaboração de teses, dissertação e trabalhos monográficos em
% geral.
%
% Este arquivo está editado na codificação de caracteres UTF-8.
%
% As referencia estão baseadas no modelo bibtex e citação em
% autor-data
%
% Este modelo foi criado por 
% 	Dr. Luís Fernando de Oliveira
% 	Professor Adjunto do Dep. de Física Aplicada e Termodinâmica
% 	Instituto de Física Armando Dias Tavares
% 	Universidade do Estado do Rio de Janeiro - UERJ
%
% A classe repUERJ.cls foi criada a partir do código original 
% disponibilizado pelo grupo CódigoLivre (coordenado por
% Gerald Weber). Foram feitas adequações para implementação das 
% normas de elaboração de teses e dissertações da UERJ.
%
% Os estilos repUERJformat.sty codificam os elementos
% pré-textuais e pós-textuais.
%
% O estilo repUERJpseudocode.sty codifica a elaboração de
% algoritmos utilizando um glossário desenvolvido por mim
% (Luís Fernando), o mesmo usado em meu curso de Física
% Computacional.
%
% Todo este material está disponível também no meu site
%      http://sites.google.com/site/deoliveiralf
%
% As normas da UERJ para elaboração de teses e dissertações 
% pode ser obtidas no documento disponível no site
%      http://www.bdtd.uerj.br/roteiro_uerj_web.pdf
%
% Agradecimentos ao NPROTEC/Rede Sirius/UERJ e à Biblioteca
% Setorial da Física.
% ---------------------------------------------------------------
% ---------------------------------------------------------------
%
% Adaptado para o Departamento de Eng. de Sistemas e Computação pelo
% professor João Araujo
\documentclass[a4paper,12pt,oneside,onecolumn,final,fleqn]{repUERJ}
% ---
% Pacotes fundamentais 
% ---
\usepackage[brazil]{babel}  % adequação para o português Brasil
\usepackage[utf8]{inputenc} % Determina a codificação utilizada
                            % (conversão automática dos acentos)
\usepackage{makeidx}        % Cria o índice
\usepackage{hyperref}       % Controla a formação do índice
\usepackage{indentfirst}    % Indenta o primeiro paragrafo de
                            % cada seção.
\usepackage{graphicx}       % Inclusão de gráficos
\usepackage{subfig}
\usepackage{multirow}
\usepackage{amsmath,amssymb}  % pacotes matemáticos
\usepackage[alf]{abntex2cite} % pacote para citacoes
\usepackage[font=default,frame=no]{repUERJformat} % pacote para as 
                                                  % normas da UERJ
% ---
% pacote auxiliar para elaboração de pseudocódigos
% este pacote pode ser retirado caso nao se planeje
% elaborar pseudocódigos
% ---
\usepackage[dots=yes]{repUERJpseudocode}
\usepackage{listings}
\lstset{language=XML, 
	keywordstyle=\color{blue},
	basicstyle=\small,
	showstringspaces=false,
	xleftmargin=0pt,
	extendedchars=true,
	inputencoding=utf8,
	frame=none,
	breaklines=true,
	emph={True,False},
	texcl=true,
	literate={á}{{\'a}}1 {à}{{\`a}}1 {ã}{{\~a}}1 {â}{{\^a}}1 {é}{{\'e}}1 {í}{{\'i}}1 {ó}{{\'o}}1 {ú}{{\'u}}1 
	{ê}{{\^e}}1 {é}{{\'e}}1 {ô}{{\^o}}1 {õ}{{\~o}}1 {ç}{{\c{c}}}1 {º}{${^\circ}$}1,, 
	commentstyle=\color{red},
	tabsize=2,
	stringstyle=\color{red},
	columns = fixed
} 

\usepackage[maxfloats=25]{morefloats}
\usepackage{array}
\usepackage{multirow}
\setlength\extrarowheight{2pt}

% ***************************************************************
% Informações de autoria e institucionais
% ***************************************************************

%----------------------------------------------------------------
% Imagens pretextuais (precisam estar no mesmo diretório deste arquivo .tex)
%----------------------------------------------------------------

\logo{logo_uerj_cinza.png}
\marcadagua{marcadagua_uerj_cinza.png}{1}{160}{255}

%----------------------------------------------------------------
% Informações da instituição
%----------------------------------------------------------------

\instituicao{Universidade do Estado do Rio de Janeiro}
            {Centro de Tecnologia e Ciências}  
            {Faculdade de Engenharia}
            {Departamento de Sistemas e Computação} 

%----------------------------------------------------------------
% Informações da autoria do documento
%----------------------------------------------------------------

% \oautor{Nóme}{Sóbrenome}
%       {Iniciais do nome} % iniciais do nome

\autor{Matheus}{Vargas}
      {M.V.} % iniciais do nome

\titulo{Adaptação do Software E-foto para Atendimento aos Requisitos de Pacotes Debian-gis}
\title{Adaptation of e-foto software to meet the requirements of debian-gis packages}

% se não for usar a quarta palavra chave, deixar o campo vazio: {}
\palavraschaves{Primeira palavra-chave}
               {Segunda palavra-chave}
               {Terceira palavra-chave}
               {Quarta palavra-chave (opcional) ou vazio}

\keywords{First keyword}
         {Second keyword}
         {Third keyword}
         {Fourth keyword or empty}

\orientador{Prof. Dr.} 
           {João}{Araujo Ribeiro} 
           {Unidade -- UERJ} 


%----------------------------------------------------------------
% Grau pretendido (Doutor, Mestre, Bacharel, Licenciado) e Curso
%----------------------------------------------------------------

\grau{Bacharel} % Doutor, Mestre, Bacharel, Licenciado
\curso{Graduação}
%\areadeconcentracao{linha de pesquisa} % opcional

%----------------------------------------------------------------
% Informações adicionais (local, data e paginas)
%----------------------------------------------------------------

\local{Rio de Janeiro} 
\data{dd}{mês}{2020} 

% ***************************************************************
% Configurações de aparência do PDF final
% ***************************************************************

% alterando o aspecto da cor azul
%\definecolor{blue}{RGB}{41,5,195}
%\definecolor{apricot}{RGB}{251,206,177}

% informações do PDF
\hypersetup{
  unicode=false,
  pdftitle={\UERJtitulo},
  pdfauthor={\UERJautor},
  pdfsubject={\UERJpreambulo},
  pdfkeywords={PALAVRAS-CHAVES:}{ \chaveA}{ \chaveB}{ \chaveC}{ \chaveD},
  pdfproducer={\packagename}, % producer of the document
  colorlinks=true,   % false: boxed links; true: colored links
  linkcolor=black,   % color of internal links blue
  citecolor=black,   % color of links to bibliography blue
  filecolor=black,   % color of file links magenta
  urlcolor=black,
  bookmarksdepth=4,
%   backref=true,
%   pagebackref=true,
%   bookmarks=true,
}

% ***************************************************************
% Índice remissivo
% ***************************************************************
%
\makeindex % compila o índice; se não for usar, comentar
%
% ***************************************************************
% Fim do preâmbulo
% ***************************************************************

%/\/\/\/\/\/\/\/\/\/\/\/\/\/\/\/\/\/\/\/\/\/\/\/\/\/\/\/\/\/
\begin{document}
%/\/\/\/\/\/\/\/\/\/\/\/\/\/\/\/\/\/\/\/\/\/\/\/\/\/\/\/\/\/

%XXXXXXXXXXXXXXXXXXXXXXXXXXXXXXXXXXXXXXXXXXXXXXXXXXXXXXXXXXXXXXXX
% ELEMENTOS PRE-TEXTUAIS
%XXXXXXXXXXXXXXXXXXXXXXXXXXXXXXXXXXXXXXXXXXXXXXXXXXXXXXXXXXXXXXXX
\frontmatter % inicia a área dos elementos pré-textuais
%XXXXXXXXXXXXXXXXXXXXXXXXXXXXXXXXXXXXXXXXXXXXXXXXXXXXXXXXXXXXXXXX

% ----------------------------------------------------------
% Capa e a folha de rosto
% ----------------------------------------------------------
%
\capa
\folhaderosto
%
% ----------------------------------------------------------
% Inserir a ficha catalográfica
% ----------------------------------------------------------
% ---
% A biblioteca deverá providenciar a ficha catalográfica. 
% Salve a ficha no formato PDF. Use o nome do arquivo PDF 
% como argumento do comando. 
% Exemplo: ficha catalográfica no arquivo 'ficha.pdf'
%     \fichacatalografica{ficha.pdf}
%
% Enquanto não possuir a ficha catalográfica, use o comando sem
% argumentos.
% ---
%
\fichacatalografica{}
%
% ----------------------------------------------------------
% Folha de aprovação
% ----------------------------------------------------------
%
\begin{folhadeaprovacao}
  \assinatura{Cargo Título Nome Completo}
             {Unidade -- Instituição}
  \assinatura{Cargo Título Nome Completo}
             {Unidade -- Instituição}
  \assinatura{Cargo Título Nome Completo}
             {Unidade -- Instituição}
  \assinatura{Cargo Título Nome Completo}
             {\UERJunidade \UERJunidadenome\ -- UERJ}
\end{folhadeaprovacao}
%
% ----------------------------------------------------------
% Dedicatória
% ----------------------------------------------------------
%
\pretextualchapter{Dedicatória}
\vfill
Texto da dedicatória (opcional).
%
% ----------------------------------------------------------
% Agradecimentos
% ----------------------------------------------------------
%
\pretextualchapter{Agradecimentos}

Texto de agradecimento (opcional).
%
% ----------------------------------------------------------
% Epigrafe (opcional)
% ----------------------------------------------------------
%
\pretextualchapter{}
  \vfill
  \begin{flushright}
 (opcional)\\
 Pensamento, reflexão\\    
    \textit{autor}
  \end{flushright}
%
% ----------------------------------------------------------
% RESUMO
% ----------------------------------------------------------
%
\pretextualchapter{Resumo}
\referencia % linha em branco depois

Texto do resumo em português.
~\\

\imprimirchaves % linha em branco antes
%
% ----------------------------------------------------------
% Abstract
% ----------------------------------------------------------
%
\pretextualchapter{Abstract}
\reference % linha em branco depois

Texto do resumo em inglês.
~\\

\printkeys % linha em branco antes
%
% ----------------------------------------------------------
% Listas de ilustrações e tabelas
% ----------------------------------------------------------
%
\listadefiguras
\listadegraficos
\listadequadros
\listadetabelas
%
% ----------------------------------------------------------
% Outras listas
% ----------------------------------------------------------
%
\listadealgoritmos % opcional
%
% ----------------------------------------------------------
% Lista de abreviaturas e siglas (opcional)
% ----------------------------------------------------------
%
\pretextualchapter{Lista de abreviaturas e siglas}
    \abreviatura{sigla 1}{por extenso}
    \abreviatura{sigla 2}{por extenso}
    \abreviatura{sigla 3}{por extenso}
%
% ----------------------------------------------------------
% Lista de simbolos (opcional)
% ----------------------------------------------------------
%
\pretextualchapter{Lista de símbolos}
    \simbolo{$simbolo 1$}{significado e/ou valor}
    \simbolo{$simbolo 2$}{significado e/ou valor}
    \simbolo{$simbolo 3$}{significado e/ou valor}
%
% ----------------------------------------------------------
% Sumario
% ----------------------------------------------------------
%
\sumario
%
%XXXXXXXXXXXXXXXXXXXXXXXXXXXXXXXXXXXXXXXXXXXXXXXXXXXXXXXXXXXXXXXX
% ELEMENTOS TEXTUAIS
%XXXXXXXXXXXXXXXXXXXXXXXXXXXXXXXXXXXXXXXXXXXXXXXXXXXXXXXXXXXXXXXX
\mainmatter % inicia a área de desenvolvimento do conteúdo
%XXXXXXXXXXXXXXXXXXXXXXXXXXXXXXXXXXXXXXXXXXXXXXXXXXXXXXXXXXXXXXXX

%================================================================
\chapter*{Introdução}
%================================================================

\begin{epigrafeonline}
\hfill Texto da epígrafe \textit{in locu}.\\
\hspace*{\fill}\textit{Autor}\\
\end{epigrafeonline}

Texto da introdução.


%================================================================
\chapter*{DESENVOLVIMENTO INICIAL}
%================================================================

Nesse capítulo será apresentado o software E-foto, suas funcionalidades e o que deve ser feito para sua obtenção e instalação. Sendo explicado que pacotes serão necessários para o funcionamento do programa após e realização do seu download.
%fiz uma introduçaõ genérica, que deve ser modificada quando o projeto for mais completo

\section{Instalação do Software E-foto nos Sistemas Operacionais Ubuntu 18.04 e 20.04}

Primeiramente o usuário deve identificar o sistema operacional em que desejará realizar a instalação do software E-foto, verificando ser Ubuntu 18.04 ou 20.04, a partir dessa verificação cabe ao usuário a escolha entre as duas formas possíveis de download do software presentes em seu website, podendo optar entre o download por SVN checkout ou em um link direto que realizará o download da sua versão .deb.

%================================================================
\chapter*{Conclusão}
%================================================================

Texto da conclusão.

%\index{Introdução!Capítulo}.

%XXXXXXXXXXXXXXXXXXXXXXXXXXXXXXXXXXXXXXXXXXXXXXXXXXXXXXXXXXXXXXXX
% ELEMENTOS POS-TEXTUAIS
%XXXXXXXXXXXXXXXXXXXXXXXXXXXXXXXXXXXXXXXXXXXXXXXXXXXXXXXXXXXXXXXX
\backmatter % inicia a área dos elementos pós-textuais
%XXXXXXXXXXXXXXXXXXXXXXXXXXXXXXXXXXXXXXXXXXXXXXXXXXXXXXXXXXXXXXXX

%===========================================================
% Referencias via BibTeX
%===========================================================

\citeoption{abnt-options4}
\bibliography{abnt-options4,bibliografia}

%===========================================================
\postextualchapter*{Glossário} % elemento opcional
%===========================================================

\definicao{termo 1}{significado}
\definicao{termo 2}{significado}
\definicao{termo 3}{significado}

%XXXXXXXXXXXXXXXXXXXXXXXXXXXXXXXXXXXXXXXXXXXXXXXXXXXXXXXXXXX
% Apêndices (opcionais)
%XXXXXXXXXXXXXXXXXXXXXXXXXXXXXXXXXXXXXXXXXXXXXXXXXXXXXXXXXXX
\appendix % inicia os apêndices
%XXXXXXXXXXXXXXXXXXXXXXXXXXXXXXXXXXXXXXXXXXXXXXXXXXXXXXXXXXX

%===========================================================

%===========================================================
\include{capAntigoProjeto}
\include{capProfile}
\include{capFicha}

%===========================================================
%===========================================================
%XXXXXXXXXXXXXXXXXXXXXXXXXXXXXXXXXXXXXXXXXXXXXXXXXXXXXXXXXXX
% Anexos (opcionais)
%XXXXXXXXXXXXXXXXXXXXXXXXXXXXXXXXXXXXXXXXXXXXXXXXXXXXXXXXXXX
\annex % inicia os anexos
%XXXXXXXXXXXXXXXXXXXXXXXXXXXXXXXXXXXXXXXXXXXXXXXXXXXXXXXXXXX

%===========================================================
\postextualchapter{Primeiro anexo}
%===========================================================

% ----------------------------------------------------------
\section{Primeira seção}
% ----------------------------------------------------------

Texto da primeira seção.

% ----------------------------------------------------------
\subsection{Primeira subseção}
% ----------------------------------------------------------

Texto da primeira subseção.

% ----------------------------------------------------------
\subsubsection{Primeira subsubseção}
% ----------------------------------------------------------

Texto da primeira subsubseção.

%===========================================================
\postextualchapter{Segundo anexo}
%===========================================================

% ----------------------------------------------------------
\section{Primeira seção}
% ----------------------------------------------------------

Texto da primeira seção.

% ----------------------------------------------------------
\subsection{Primeira subseção}
% ----------------------------------------------------------

Texto da primeira subseção.

% ----------------------------------------------------------
\subsubsection{Primeira subsubseção}
% ----------------------------------------------------------

Texto da primeira subsubseção.

%/\/\/\/\/\/\/\/\/\/\/\/\/\/\/\/\/\/\/\/\/\/\/\/\/\/\/\/\/\/
\end{document}
%/\/\/\/\/\/\/\/\/\/\/\/\/\/\/\/\/\/\/\/\/\/\/\/\/\/\/\/\/\/
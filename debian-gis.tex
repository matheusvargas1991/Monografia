%================================================================
\chapter{Debian-GIS}
%================================================================
\section{O que é o Debian-GIS}

O \textbf{Debian-GIS} é um \textit{Debian Pure Blend}, o que significa ser uma parte do Debian nos quais seus pacotes estão agrupados e disponíveis para um determinado grupo de usuários que possuem necessidades especiais, tornando desnecessário o acesso a todos os pacotes Debian disponíveis. O \textit{Debian-GIS} tem como objetivo atender os usuários interessados em um Sistema de Informação Geográfica, ou seja, visa satisfazer as necessidades de usuários que trabalham com mapas, sensoriamento remoto e observação da Terra e dispõe de uma lista de programas GIS selecionados para o Debian.

A equipe \textit{Debian-GIS} também atua em conjunto para ajudar a manter o software GIS atual em derivados Debian, como \textbf{UbuntuGIS}, que reempacota os pacotes do Debian-GIS nas diferentes distribuições do Ubuntu, e \textbf{OSGeo-Live} fornece aplicativos pré-configurados para uma variedade de casos de uso geoespacial, incluindo armazenamento, publicação, visualização, análise e manipulação de dados. Ele também contém conjuntos de dados de amostra e documentação.

De acordo com o site do projeto\footnote{\url{https://debian-gis-team.pages.debian.net/policy/packaging.html}} \cite{bib:gis}:

\begin{quote}
	``O Debian GIS se orgulha de que derivados estão lucrando com o trabalho dentro do Debian e está tentando estabelecer conexões fortes com esses derivados. Com o UbuntuGIS, a conexão é tão forte que um fluxo de trabalho comum foi criado, onde os desenvolvedores do UbuntuGIS estão injetando seu pacote diretamente no sistema de controle de versão do Debian-GIS. Para ter certeza de que não haverá conflitos com as revisões do Debian, deve-se prestar atenção à numeração das revisões.''
\end{quote}

\subsection{Tasks}

De acordo com o site do Debiena-GIS\footnote{\url{https://debian-gis-team.pages.debian.net/policy/tasks.html}} \cite{bib:gis}:

\begin{quote}
	``O Debian GIS Debian Pure Blend é organizado por tasks, que agrupam pacotes em torno de temas amplos como serviços web e estação de trabalho. As tasks listam programas que já estão empacotados no Debian assim como pacotes em preparação. Os arquivos de tasks não são hospedados nos repositórios Debian-GIS, mas no repositório Debian Blends, em uma área de trabalho de computador normal, o usuário provavelmente desejará instalar pelo menos a task da estação de trabalho que contém os aplicativos GIS mais usados. Os meta-pacotes do Debian GIS Blend podem ser instalados em uma instalação normal do Debian. Como isso pode ser feito depende de qual versão do Debian você está executando.''  
\end{quote}

Estes são os metapacotes citados acima que podem ser instalados usando o comando apt-get install (nome do metapacote):

\begin{itemize}
	\item gis-data - dados Debian GIS
	\item gis-gps - programas relacionados a GPS
	\item gis-osm - Programas relacionados ao OpenStreetMap
	\item gis-remotesensing - Sensoriamento remoto e observação da Terra
	\item gis-statistics - Estatísticas com dados geográficos
	\item gis-web - Apresentar informações geográficas via web mapserver
	\item gis-workstation - estação de trabalho de Sistemas de Informação Geográfica (GIS)
	\item gis-devel - pacotes de desenvolvimento GIS

\end{itemize}

\section{Sistemas Debian}
% falta revisar
O sistema Debian ou Debian GNU/Linux é um sistema operacional que faz parte do projeto Debian, que é uma organização de pessoas com interesse em comum de criar um sistema operacional livre. Nesse projeto os membros são voluntários, e o sistema operacional Debian, que é um conjunto de programas básicos e utilitários que fazem o computador funcionar, pode ser obtido gratuitamente na web.

O sistema Debian é muito conhecido devido ao seu poderoso sistema de gerenciamento de pacotes (APT, \textit{Advanced Packaging Tool} ou Ferramenta de Empacotamento Avançada) que permite a instalação de novos pacotes, além da atualização e remoção dos pacotes antigos de forma limpa e relativamente fácil. O Debian possui acesso a repositórios online com uma quantidade enorme de pacotes, sendo oficialmente apenas programas de software livre. Programas ``\textit{não-livres}'' também` podem ser baixados e instalados, caso seja necessário.

De acordo com o próprio site oficial do Debian\footnote{\url{www.debian.org}}, o Debian é uma organização totalmente voluntária dedicada a desenvolver software livre e promover os ideais da comunidade de Software Livre. O projeto Debian se iniciou em 1993, quando Ian Murdock ofereceu um convite livre a desenvolvedores de software livre para contribuir com uma distribuição completa e coerente, baseada no \textit{kernel} do Linux relativamente novo. O nome “Debian” vem da junção do nome do principal fundador, Ian, com o de sua esposa, Debra. Os Desenvolvedores Debian estão envolvidos em uma variedades de atividades, incluindo Web e FTP, administração do site, design gráfico, análise legal de licenças de software, escrevendo documentação, e é claro, mantendo pacotes de software.

\subsection{Vantagens da utilização do Debian}
A página oficial\footnote{\url{www.debian.org/intro/why\_debian.pt.html}} lista o motivo de escolher o \textit{Debian} para utilizar em uma máquina e entre suas principais vantagens estão:

\begin{itemize}
	\item Velocidade na resolução de dúvidas em sua lista de discussão;
	\item O sistema é mantido por seus próprios usuários;
	\item Existem muitas empresas e pessoas que usam o Debian\footnote{\url{www.debian.org/users}};
	\item O \textit{Dpkg} é um dos melhores sistema de empacotamento atualmente, cansativos testes são feitos para incluir programas novos aos sistema \textit{Debian stable};
	\item Instalação melhorada periodicamente, já que antes a dificuldade de instalação era um problema bem comentado pelos interessados em migrar para o Debian;
	\item São mais de 59.000 programas diferentes;
	\item Atualização do sistema simplificado com o APT;
	\item Suporte para múltiplas arquiteturas (alpha, amd64, armel, hppa, i386, ia64, mips, mipsel, powerpc, s390, e sparc);
	\item Estável, leve e rápido, com \textit{drivers} escritos pelos usuários, sem dependência dos fabricantes.
\end{itemize}

Mas como nada é perfeito, o Debian também tem alguns pontos de reclamação que ainda estão sendo otimizados, que são:

\begin{itemize}
	\item Nem todo hardware é suportado;
	\item O Debian é difícil de configurar;
	\item Falta de software comercial popular.

\end{itemize}

\subsection{Derivados do Debian}

De acordo com o site oficial\footnote{\url{www.debian.org/derivatives/index.pt.html}}: 

\begin{quote}
	``Existem várias distribuições baseadas no Debian. Algumas pessoas podem querer dar uma olhada nessas distribuições, além dos lançamentos oficiais do Debian. Um derivado do Debian é uma distribuição baseada no trabalho realizado no Debian, mas com sua própria identidade, objetivos e público-alvo, e é criado por uma entidade que é independente do Debian. Os derivados modificam o Debian para atingir os objetivos que eles mesmos estabeleceram. O Debian dá boas-vindas e encoraja organizações que desejam desenvolver novas distribuições baseadas no Debian. No espírito do contrato social do Debian, é esperado que os derivados contribuam com seu trabalho para o Debian e os projetos dos autores originais, para que todos possam se beneficiar de suas melhorias.''
\end{quote}

O direcionamento do público de certo nicho para o derivado Debian que contém usuários que partilham de objetivos semelhantes, ajuda a manter a satisfação do usuário com o Debian como um todo, já que seu acesso a recursos será direcionado a realização das tarefas de seu nicho com mais facilidade e sem a necessidade de ter que conhecer e dominar todos os recursos do Debian. E dado esse aumento na satisfação do publico de diferentes nichos, é de se esperar que a comunidade Debian cresça como um todo.

\section{Requisitos para um software ser aceito no Debian-GIS}

No geral para um pacote ser aceito no \textit{Debian-GIS} as regras são basicamente as mesmas para que o pacote seja aceito no Debian, pois é aconselhável seguir o caminho normal para a aceitação de pacotes Debian em pacotes \textit{Debian-Gis}. Porém existem algumas edições no arquivo \textit{control} do repositório Debian do software empacotado que devem ser realizadas e que são específicas para o Debian-Gis, de acordo com a política oficial Debian-GIS\footnote{ \url{https://debian-gis-team.pages.debian.net/policy/policy.html}}.

\subsection{Mudanças no arquivo Debian/control}
Um pacote é uma versão compactada de um software ou de uma biblioteca e são usados para a distribuição dos mesmos. Cada pacote presente em Sistemas Debian contém o arquivo control dentro de um diretório Debian e esse arquivo é preenchido com informações vitais sobre do que é formado o pacote, tanto tratando do pacote fonte quanto dos binários construídos.

De acordo com a Política Oficial do Debian-GIS, as seguintes mudanças devem ser feitas obrigatoriamente em campos do arquivo Debian/control para que o pacote seja aceito. 

\begin{verbatim}
Section (Seção): Deve ser “science” para o pacote de origem.
Priority (Prioridade): Deve ser “optional” a menos que proibido pela
política Debian.
Maintainer (Mantenedor): O mantenedor deve ser Debian GIS Project
<pkg-grass-devel@lists.alioth.debian.org>. O usuário pode se
inscrever também nesta lista caso queira participar.
Uploaders: O usuário deve se increver como um uploader quando
tiver um interesse significativo em um pacote.
\end{verbatim}

\subsection{Busca por um patrocinador}

Para quem não é um desenvolvedor oficial do Debian, para que o pacote seja posto no Debian é necessária a busca por um sponsor que revisará esse pacote e realizará a operação de \textit{upload}. Existem vários \textit{Debian Developers} na equipe \textit{Debian-GIS}, mas é comum que eles estejam ocupados e acabem não conseguindo responder a todos os pedidos de patrocínio enviados para a Lista de Discussão dos Desenvolvedores \textit{Debian-GIS}. Portanto, é recomendado seguir o processo normal do Debian que é enviar um relatório de bug de Solicitação de Patrocínio (RFS). 

O site \textit{Debian Mentors} é o melhor lugar para enviar seu pacote para atrair patrocinadores, o site também fornece um modelo para o seu relatório de bug RFS. Na política do \textit{Debian-GIS} tem o aconselhamento de como preencher alguns itens desse relatório de erro para que ele se destaque dos demais pacotes do Debian na busca pelo patrocinador.

O relatório de bug RFS também deve ser copiado para a lista de discussão dos desenvolvedores GIS do Debian. A melhor maneira de fazer isso é usar o cabeçalho\textit{ X-Debbugs-CC} no relatório de bug. Adicione \textit{X-Debbugs-CC: pkg-grass-devel@lists.alioth.debian.org} ao cabeçalho de e-mail da mensagem ou use a opção \textbf{s} no utilitário reportbug.

Seguindo esses passos no momento da criação do relatório de bug, que é utilizado para pedir um patrocinador para o novo pacote criado, chamará atenção de desenvolvedores que fazem parte do \textit{Debian-GIS} especificamente e eles poderão ajudar a avaliação do pacote para ver se ele é feito corretamente e digno de fazer parte do Debian e \textit{Debian-GIS}. Já que o relatorio de bug estará preenchido de acordo com o que é indicado pela política do \textit{Debian-GIS}.
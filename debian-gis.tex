%================================================================
\chapter{Debian-GIS}
%================================================================
\section{O que é o Debian-GIS}

O Debian-GIS é um Debian Pure Blend, o que significa ser uma parte do Debian onde seus pacotes estão agrupados e disponíveis para um determinado grupo de usuários que possuem necessidades especiais tornando desnecessário o acesso a todos os pacotes Debian disponíveis. O Debian-GIS tem como objetivo atender os usuários interessados em um Sistema de Informação Geográfica, ou seja, visa satisfazer as necessidades de usuários que trabalham com mapas, sensoriamento remoto e observação da Terra e dispõe de uma lista de programas GIS selecionados para o Debian.

\section{Sistemas Debian}

O sistema Debian ou Debian GNU/Linux é um sistema operacional que faz parte do projeto Debian, que é uma organização de pessoas com interesse em comum de criar um sistema operacional livre. Nesse projeto os membros são voluntários, e o sistema operacional Debian, que é um conjunto de programas básicos e utilitários que fazem o computador funcionar, pode ser adquirido gratuitamente na web. O sistema Debian é muito conhecido devido ao seu poderoso sistema de gerenciamento de pacotes (APT, Advanced Packaging Tool ou Ferramenta de Empacotamento Avançada) que permite a instalação de novos pacotes, além da atualização e remoção dos pacotes antigos de forma limpa e relativamente fácil. O Debian possui acesso a repositórios online com uma quantidade enorme de pacotes, sendo oficialmente apenas softwares livres. Softwares não-livres também podem ser baixados e instalados caso seja necessário.

\section{Requisitos para um software ser aceito no Debian-GIS}
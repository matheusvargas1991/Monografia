%================================================================
\chapter{O programa E-foto}
%================================================================

Nesse capítulo será apresentado o software E-foto, suas funcionalidades e o que deve ser feito para sua obtenção e instalação. Sendo explicado que pacotes serão necessários para o funcionamento do programa após e realização do seu download.
%fiz uma introduçaõ genérica, que deve ser modificada quando o projeto for mais completo

\section{O que é o E-foto}

O E-foto é um software live para fotogrametria digital que é desenvolvido pelo laboratório de fotogrametria da Universidade do Estado do Rio de Janeiro desde 2004 e tem como objetivo, além da criação de um software inteiramente funcional (uma estação fotogramétrica gratuita), levar aos alunos o conhecimento de forma gratuita sobre fotogrametria digital, sendo de forma didática ou ate mesmo na prática por meio de acesso ao código, uso da plataforma e até criação de novos módulos para o software. A fotogrametria é a obtenção de informações confiáveis por meio de imagens obtidas por sensores, sendo no caso da fotogrametria digital a entrada de dados são imagens digitais provenientes de máquinas digitais ou pela digitalização de imagens obtidas de forma analógica.

\section{Instalação do E-foto}
\subsection{Em Sistemas Linux}
\subsection{Em Sistemas Windows}

\subsection{Compilação do E-foto a partir dos fontes}
\subsection{Em Sistemas Linux}
% é necessário que você faça um verdadeiro passo a passo.
% detalhe cada passo. Pense que você está ensinando o seu pai a compilar o e-foto. Seja minucioso e sem ambiguidades. 

Primeiramente o usuário deve identificar o sistema operacional em que desejará realizar a compilação do software E-foto, verificando ser Ubuntu 18.04 ou 20.04, a partir dessa verificação cabe ao usuário a escolha entre as duas formas possíveis de download do software presentes em seu web site http://www.efoto.eng.uerj.br/download/latest-version, podendo optar entre o download por SVN checkout que para isso o usuário deve possuir o Tortoise SVN, que pode ser adquirido gratuitamente por esse link https://tortoisesvn.net/downloads.html e utilizar o https://svn.code.sf.net/p/e-foto/code na opção de checkout para realizar o download ou em um link direto que realizará o download da sua versão .deb. Após a realização com opção desejada o usuário deve ficar atento aos pacotes necessários em seu ambiente para que o software E-foto possa ser instalado e ter seu funcionamento sem erros. Para a instalação dos pacotes o usuário deve buscar o ícone do terminal na sua barra de ferramentas e de preferência usar o comando sudo apt-get update que atualizará a lista de pacotes disponíveis para a busca da ferramenta de empacotamento. Um desses pacotes que são necessários é o libgdal.dev, que deve ser instalado a partir do seguinte comando no  terminal sudo apt-get install libgdal-dev, onde GDAL é uma biblioteca de tradução para formatos geoespaciais, outros pacotes de grande importância são os de dependência do Qt 5 que são os pacotes build-essential que deve ser instalado com o comando sudo apt-get install build-essential no terminal, que proverá os compiladores necessários, e libx11-xcb-dev e libglu1-mesa-dev, que serão necessários caso o usuário não tenha nenhum driver XCB e OpenGl instalados por default, já que o Qt espera a instalação dos mesmos. Por último os pacotes para a instalação do Qt 5 propriamente dito que são os pacotes qt5-default, instalado através do comando no terminal sudo apt install qt5-default, e qt5-qmake, instalado através do comando sudo apt-get install -y qt5-qmake, que permitirão o uso do Qt 5 e consequentemente a instalação do Software E-foto, para executar o software via terminal, após a realização de todos os passos necessários, o usuário deve usar o comando ./build/bin/e-foto no terminal. Existe outra opção para a instalação do software E-foto, que é usando um software chamado QtCreator que, caso seja a opção escolhida, deve ser instalado após todo o procedimento de instalação do Qt 5 e seus pacotes de dependências. Com o QtCreator, o usuário deve escolher o kit correspondente ao Qt 5 no momento em que for perguntado sobre qual a configuração escolhida para o projeto que está sendo aberto e após essa etapa basta compilar e executar o projeto do E-foto que o mesmo funcionará perfeitamente.
No caso da opção escolhida de download tenha sido a versão .deb, o usuário necessita instalar os pacotes de dependência do Qt 5 e depois via terminal ir até o caminho onde se encontra o arquivo .deb usando o comando cd /lugardoarquivo e usar o comando sudo dpkg -i nomedoarquivo.deb, para que o dpkg (programa de gerenciamento de pacotes das versões linus Debian) realizará a instalação do pacote Debian do E-foto.

\subsection{Em Sistemas Windows}

Em sistemas Windows o usuário deve se certificar de estar utilizando um sistema de 64-bits pois não existe versão do E-foto para sistemas de 32-bits, para realizar esse procedimento de verificação de qual é a versão do sistema que está sendo utilizada basta digitar "informações do sistema" no menu iniciar, no menu Resumo do Sistema existe a opção Versão do Sistema que dará ao usuário essa informação. O próximo passo é o download dos pacotes binários do Gdal no Windows que pode seer realizado através desse link https://repo.msys2.org/distrib/x86_64/msys2-x86_64-20200720.exe, após a o download e da instalação do MSYS o usuário deve executar o que vai abrir o terminal do próprio MSYS que contém uma versão portada do gerenciador de pacotes Pacman, nessa ultima versão do Pacman deve ser digitado no terminal o código pacman -Syuu, que gerará um série de instruções a serem seguidas até que o usuário possa repetir o código e receber a mensagem de que nada necessita ser atualizado. Com o ambiente atualizado o usuário deve digitar o seguinte comando pacman -S mingw64/mingw-w64-x86_64-gdal, que realizará finalmente o binário da GDAL. Após todo esse procedimento o usuário pode consultar no terminal do MSYS através do código gdalinfo --version se o gdal realmente foi instalado e a versão em que ele está disponível. Para seguir, o usuário deve possuir o Tortoise SVN que pode ser adquirido gratuitamente por esse link https://tortoisesvn.net/downloads.html, após a instalação do SVN o usuário deve buscar no programa a opção de SVN Checkout e no espaço para colar uma URL colocar https://svn.code.sf.net/p/e-foto/code, que deve gerar o download de todo o código fonte do E-foto. O Usuário deve, por ultimo, realizar o download do Qt 5 e do Qtcreator no website https://www.qt.io/download, deixando claro que na parte da instalação deve ser instalado apenas o mingw 64-bits, e o Qt 5 referente a esse sistema. Com os arquivos do código fonte do E-foto disponíveis o usuário deve procurar no caminho e-foto-code/branches/e-foto-trunk-candidate por um arquivo chamado e-foto.pro. Quando abrir o projeto no Qtcreator será necessário configurar o que deverá ser usado, começando pela escolha do kit que será usado que é o Qt 5 e o mingw 64-bits. Após o projeto estar aberto, o usuário deve ir na guia project e desmarcar a opção shadow build, e depois procurar na aba build a opção build E-foto, quando acabar a compilação é só buscar a opção Run e o E-foto vai abrir pronto para o uso.
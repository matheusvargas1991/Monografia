%================================================================
\chapter{O programa E-foto}
%================================================================

Nesse capítulo será apresentado o software E-foto, suas funcionalidades e o que deve ser feito para sua obtenção e instalação. Sendo explicado que pacotes serão necessários para o funcionamento do programa após e realização do seu download.
%fiz uma introduçaõ genérica, que deve ser modificada quando o projeto for mais completo

\section{O que é o E-foto}

\section{Instalação do E-foto}
\subsection{Em Sistemas Linux}
\subsection{Em Sistemas Windows}

\subsection{Compilação do E-foto a partir dos fontes}
\subsection{Em Sistemas Linux}
% é necessário que você faça um verdadeiro passo a passo.
% detalhe cada passo. Pense que você está ensinando o seu pai a compilar o e-foto. Seja minucioso e sem ambiguidades. 

Primeiramente o usuário deve identificar o sistema operacional em que desejará realizar a compilação do software E-foto, verificando ser Ubuntu 18.04 ou 20.04, a partir dessa verificação cabe ao usuário a escolha entre as duas formas possíveis de download do software presentes em seu website, podendo optar entre o download por SVN checkout ou em um link direto que realizará o download da sua versão .deb. Após a realização com opção desejada o usuário deve ficar atento aos pacotes que devem necessariamente ser instalados para que o software E-foto possa ser instalado e ter seu funcionamento sem erros. Um desses pacotes são o libgdal.dev, onde GDAL é uma biblioteca de tradução para formatos geoespaciais, outros pacotes de grande importância são os de dependência do qt5 que são os pacotes build-essential, que proverá os compiladores necessários, e libx11-xcb-dev e libglu1-mesa-dev que serão necessários caso o usuário não tenha nenhum driver XCB e OpenGl instalados por default já que o Qt espera a instalação dos mesmos. Por último os pacotes para a instalação do Qt5 propriamente dito que são os pacotes qt5-default e qt5-qmake que permitirão o uso do qt5 e consequentemente a instalação do Software E-foto via terminal. Existe outra opção de instalação do software E-foto que é usando um software chamado QtCreator que, caso seja a opção escolhida, deve ser instalado após todo o procedimento de instalação do qt5 e suas dependências, com o QtCreator, o usuário deve escolher o kit correspondente ao qt5 no momento em que for perguntado qual a configuração escolhida para o projeto que está sendo aberto e após essa etapa basta compilar e executar o projeto do E-foto que o mesmo funcionará perfeitamente.

\subsection{Em Sistemas Windows}